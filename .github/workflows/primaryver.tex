\documentclass[a4paper]{book}

\usepackage[english,russian]{babel}
\usepackage[utf8x]{inputenc}
\usepackage[T1]{fontenc}
\usepackage{amsmath,amssymb,amsthm}
\usepackage{euscript}
\usepackage{mathrsfs}
\usepackage{icomma}
\usepackage{amsfonts}
\usepackage{amscd}
\usepackage[all]{xy}
\usepackage[dvips]{graphicx}
\graphicspath{{noiseimages/}}
\usepackage{graphicx}


\DeclareMathOperator{\sgn}{\mathop{sgn}}



\newcommand*{\hm}[1]{#1\nobreak\discretionary{}
{\hbox{$\mathsurround=0pt #1$}}{}}


\author{О.Арсений}
\title{Высшая математика для психологов}
\date{\today}

\begin{document} 
	
	\maketitle
	Необходимость понимания математических предпосылок реальности самоочевидна. Доказательство этого утверждения предоставляется читателю. 
	
\chapter{Ликбез}

\begin{flushright}
	Математика содержит в себе черты волевой деятельности, умозри-
	тельного рассуждения и стремления к эстетическому совершенству. Р.Курант
\end{flushright}

\section{Немного истории математики}



\section{Системы счисления}
Системой счисления называют определённый способ представления чисел в знаках. Всякий элемент произвольного числового множества имеет уникальное представление в системе счисления. 

Системы счисления бывают: 
\begin{itemize}
	\item Позиционные (англ. positional system);
	\item Непозиционные; 
	\item Смешанные; 
\end{itemize}

Важно понимать, что представление числа и определение понятия \textit{число} -- разные вещи. Как бы не было представлено одно число, какую бы систему счисления мы не использовали для его представления в знаках, внутренняя суть числа всегда будет единственна. Поэтому важно разводить число, записью которого является $9$ в $10$-ричной системе счисления и цифру $9$, которая используется для его (числа) записи. 

В позиционных системах счисления одна и та же цифра в записи числа имеет различные значения в зависимости от разряда, то есть места в числе, где она расположена. Изобретение позиционной нумерации, основанной на поместном значении цифр, приписывается шумерам и вавилонянам; развита была такая нумерация индусами и имела неоценимые последствия в истории человеческой цивилизации. К числу таких систем относится современная десятичная система счисления, возникновение которой связано со счётом на пальцах. В средневековой Европе она появилась через итальянских купцов, в свою очередь заимствовавших её у арабов.

Обычно под позиционной системой счисления понимается некая $b$-ичная система счисления, которая определяется числом $b\in\mathbb{N}$. Такое $b$ называется \textit{основанием} системы счисления. Основание, по сути, является числом различных используемых цифр в записях произвольных чисел. При этом само число-основание в "своей"  системе счисления не имеет собственной цифры. 

Целое число представляется в $b$-ичной системе счисления в виде \textit{конечной линейной комбинации} степеней числа $b$:

\begin{equation}
x = \sum_{k=0}^{n-1} a_k b^k 
\end{equation}

Где $a_k$ -- натуральные числа, удовлетворяющие неравенству $0 \leq a_k \leq (b-1)$. Каждая степень $b^k$  в такой записи называется весовым коэффициентом разряда. Старшинство разрядов и соответствующих им цифр определяется значением показателя  $k$ (номером разряда). Обычно в записи ненулевых чисел начальные нули опускаются.

Если не возникает разночтений (например, когда все цифры представляются в виде уникальных письменных знаков), число  $x$ записывают в виде последовательности его  $b$-ичных цифр, перечисляемых по убыванию старшинства разрядов слева направо:

\begin{equation}
x = a_{n-1} a_{n-2} ... a_0
\end{equation}

\textbf{Пример}. Число $467,89$ представленное в десятичной системе счисления, раскладывается в следующую конечную линейную комбинацию: $467,89 = 4\cdot 10^2 + 6\cdot 10^1 + 7\cdot 10^0 + 8\cdot 10^{-1} + 9\cdot 10^{-2}$. 

\section{Что такое математическое доказательство}


\chapter{Теория множеств}
\section{Основные понятия и обозначения}
Этот раздел обязателен для освоения. 

Классическое определение множества Г.Кантора: "Под множеством мы понимаем любое соединение $M$ определенных различимых объектов нашего умозрения или нашей мысли (которые будут называться элементами $M$) в единое целое". 

Р.Дедекинд писал, что "множество $S$ полностью определено тогда и только тогда, когда относительно всякой вещи известно, является она элементом множества $S$ или нет".

Множества состоят из элементов. Запись вида $a\in A$ означает, что $a$ является элементом множества $A$. Через $\{a,b,c,d\}$ обозначают элементы некоторого четырёхэлементного множества $A$, которое содержит только элементы $a,b,c,d$ и не содержит других элементов. Важно понимать, что запись $\{a,b,b,c,c,d\}$ и запись $\{a,b,c,d\}$ обозначают одно и то же множество. Отношение принадлежности не является транзитивным: из $a \in b \wedge b \in c \nRightarrow a \in c$. То есть, если $\varnothing \in \{\varnothing\} \wedge \{\varnothing\} \in \{\{\varnothing\}\} \Rightarrow \varnothing \notin \{\{\varnothing\}\}$. Как заметил О.Шпенглер, уяснение разницы между вещью $x$ и множеством $\{x\}$, единственным элементом которого является эта вещь, представляет серьёзные психологические труднодсти для "детей, народных масс и философов". 

\textbf{Определение}. Множество $A$ называется подмножеством множества $B$, что обозначается $A\subset B$, когда всякий элемент множества $A$ является элементом множества $B$. Что можно записать так: $\forall a \in A \Rightarrow a \in B$.

\textbf{Определение}. Множества $A$ и $B$ называются равными, если элементы, принадлежащие им, совпадают. Это можно записать так: $A\subset B \wedge B\subset A$. 

\textbf{Определение}. Множество не содержащие ни одного элемента называется пустым множеством. Такое множество содержится как подмножество в любом множестве (даже в самом себе). Пустое множество обозначается одним из следующих символов: $\varnothing$, $\emptyset$ или $\{\}$. 
\\$\varnothing\subset\varnothing$ $\wedge$ $\varnothing \supset \varnothing$. $\varnothing$ - это совокупность тех элементов всякого множества, которые ему не принадлежат. То есть, для произвольного свойства $P$ и произвольного множества $A$, верным будет утверждение, что существует такое подмножество в этом множестве, элементам которого данное свойство присуще. 

\textbf{Определение}. Пересечением двух множеств $A$ и $B$ называется такое множество $C = A\cap B$, состоящие из всех тех элементов, которые принадлежат обоим множествам. Тогда: 

\begin{equation}
C = \{c \mid c\in A \wedge c\in B\}
\end{equation} 

("$c \mid$" читается: множество таких $c$ что...).

\textbf{Определение}. Объединением двух множеств $A$ и $B$ называется такое множество $C = A\cup B$, состоящие из всех элементов, принадлежащих хотя бы одному из множеств:

\begin{equation}
C = \{c \mid c\in A \vee c\in B\}
\end{equation} 

\begin{equation}
(A\cap B)\cup (B \cap C) = (A \cup C) \cap B
\end{equation}

\begin{equation}
(A\cup B)\cap (B \cup C) = (A \cap C) \cup B
\end{equation}

\textbf{Определение}. Разностью двух множеств $A$ и $B$ называется такое множество $C = A\setminus B$, состоящие из элементов, принадлежащих только множеству $A$, но не принадлежащих множеству $B$:

\begin{equation} 
C = \{c \mid c\in A \wedge c\notin B\}
\end{equation} 

Если $A\cap B = \varnothing$, то $A\setminus B = A$. Если $B \subset A$, то $A\setminus B$ называется дополнением множества $B$ до множества $A$. 

\begin{equation} \label{proof4}
X \setminus \bigcap_i A_i = \bigcup_i(X\setminus A_i)
\end{equation}

\begin{equation} \label{proof2}
X \setminus \bigcup_i A_i = \bigcap_i(X\setminus A_i)
\end{equation}

В первом тождестве слева мы рассматриваем все такие элементы множества $X$, которые не содержатся хотя бы в одном из множеств семейства $A_i$, а, по сути, то же самое. Ведь это все такие элементы множества $X$, которые лежат в дополнении ко всякому множеству $A_i$, то есть, другими словами, они не содержатся хотя бы в одном из множеств вида $A_i$. 

Во втором тождестве слева мы рассматриваем все такие элементы множества $X$, которые не содержатся ни в каком из множеств $A_i$, то есть, ни в одном из них. Слева же мы рассматриваем все те элементы множества $X$, которые не лежат ни в одном из множеств семейства, то есть, лежат во всяком дополнении любого из множеств вида $A_i$ до $X$. 

\textbf{Определение}. Симметрической разностью двух множеств $A$ и $B$ называется такое множество $C = A\bigtriangleup B$, состоящие из элементов, принадлежащих только множеству $A$ и только множеству $B$, но не их пересечению. 

\begin{equation}
C = (A\setminus B)\cup(B\setminus A) = (A\cup B)\setminus (A\cap B)
\end{equation}

Если $A\cap B = \varnothing$, то $A\bigtriangleup B = A \cup B$. Что самоочевидно. Эта операция ассоциативна и коммутативна, что будет немаловажно дальше: $A\bigtriangleup B = B \bigtriangleup A$ и $(A\bigtriangleup B)\bigtriangleup C = A\bigtriangleup (B\bigtriangleup C).$

Множества $\varnothing, \{\varnothing\}, \{\varnothing, \{\varnothing\}\}$ являются совершенно разными. Первое множество пусто и не содержит элементов вообще. Второе множество содержит один элемент - пустое множество. Третье множество содержит два элемента - пустое множество и одноэлементное множество или синглетон $\{\varnothing\}$. Важно понимать, что эти два последних множества содержат $\varnothing$ как элемент, а не только как подмножество. Множества, содержащие в качестве своих элементов другие множества, называются в теории множеств классами или семействами множеств. 

\section{Примеры множеств}

Этот раздел обязателен для освоения, но при первом прочтении учебника его можно лишь просмотреть. Некоторые понятия, такие как: поле, подполе, кольцо, подкольцо и так далее, ещё не были строго определены.

Наиболее часто нам придётся сталкиваться с различными числовыми множествами. Вот некоторые из них: 
\begin{itemize}
	\item Множество натуральных чисел. Оно обозначается вот так: $\mathbb{N}$. Это счётное множество, что означает, что число его элементов бесконечно, но может быть проиндексировано (пронумеровано) натуральными числами. Это дискретное множество - между всякими двумя натуральными числами находится конечное число натуральных чисел. 
	\item Множество целых чисел. Оно обозначается вот так: $\mathbb{Z}$. Множество целых чисел больших нуля можно записать как $\mathbb{Z}_0 = \mathbb{N}\cup \{0\}$. $\mathbb{Z}$ образует коммутативное кольцо с единицей. Его подмножество - множество четных чисел, образующее коммутативное подкольцо без единицы.Записать это можно так: $ \mathbb{Z} = \{z\mid z\mod 2 \equiv 0\}$.
	\item Множество рациональных чисел. Оно обозначается вот так: $\mathbb{Q}$. Это счётное множество. Каждое рациональное число - пара двух чисел (целого и натурального). $\mathbb{Q} = \{q \mid q = \frac {z}{n}, где z \in \mathbb{Z} \wedge n \in \mathbb{N} \}$. Множество $\mathbb{Q}$ образует поле. Это уже не дискретное множество. 
	\item Множество действительных или вещественных чисел. Оно обозначается вот так: $\mathbb{R}$. Это несчётное множество, получаемое путём объединения счетного множества рациональных чисел $\mathbb{Q}$ и несчетного множества иррациональных чисел $\mathbb{I}$: $\mathbb{R} = \mathbb{Q} \cup \mathbb{I}$. Множество $\mathbb{R}$ образует поле, при этом $\mathbb{R}$ является полным множеством, что означает, что между любыми двумя элементами этого множества находится столько же элементов, сколько и во всём множестве $\mathbb{R}$.
	\item Множество комплексных чисел. Оно обозначается вот так: $\mathbb{C}$. Комплексные числа (как и рациональные) представлены парой чисел $a + bi$, где $a,b \in \mathbb{R}$, а $i = \sqrt{-1}$. $i$ называется мнимой единицей. Комплексные числа тоже образуют поле, но их уже нельзя линейно упорядочить (как действительные числа). Поле $\mathbb{C}$ содержит поле $\mathbb{R}$ как подполе. Если существует такое подполе $\mathbb{K} \subset \mathbb{C}$, в котором содержится $i \in \mathbb{K}$ и подполе $\mathbb{R} \subset \mathbb{K}$, то $\mathbb{K} = \mathbb{C}$.
	
\end{itemize}
 Существуют и другие числовые множества, но нам они не понадобятся в этом курсе. 
 
\section{Мощность множества}
 Этот раздел обязателен для освоения. 
 
 Мощностью произвольного множества $A$ называется число его элементов и обозначается это $ | A |$ (на самом деле, это обобщение понятия числа элементов множества, которое, по сути, осмысленно только для конечных множеств). Ранее мы уже сталкивались с тремя основными видами мощностей: конечные, счётные и несчётные. Внутри класса конечных множеств существуют свои градации по мощности, например, множество $A = \{a,b,c\}$ и $B = \{a,b,c,d\}$ не равномощны. То есть, $| A | < | B |$. Однако с бесконечными множествами всё иначе. Если мы возьмём два произвольных счетных множества, например, $\mathbb{N}$ и $\mathbb{Q}$, они всё равно будут равномощны, хотя первое является подмножеством второго, но второе -- не подмножество первого. 
 Между равномощными множествами можно установить взаимно-однозначное соответствие (или биекцию).
 
 \textbf{Теорема 1. Формула включений и исключений.} 
\begin{equation}
 | A \cup B | = | A| + | B| - | A \cap B |
\end{equation}

\begin{equation}
| A \cup B \cup C| = | A| + | B| + | C| - | A \cap B| - | A \cap C| - | B \cap C| + | A \cap B \cap C|
\end{equation}

\begin{equation}
| A_1 \cup A_2 ... \cup A_n | = \sum_{i} | A_i| - \sum_{i<j} | A_i \cap A_j | + \sum_{i<j<k}  | A_i \cap A_j \cap A_k | - ... +(1)^{n-1}| A_1 \cap A_2 \cap ... \cap A_n |
\end{equation}

Доказательство (через характеристические функции): Допустим, существует некое множество $U$, подмножествами которого являются множества $A,B,...$. 

Здесь мы будем использовать понятие функции, которое определим только чуть позже. 

Характеристической функцией множества $A \subset U$ называют такую функцию $\chi_A$, которая равна 1 на всех элементах множества $A$ и нулю на всех остальных элементах множества $U$, то есть на $U\setminus A$. В таком случае, пересечению двух подмножества будет соответствовать произведению их характеристических функций: 

\begin{equation}
Для \forall u \in A\cap B \Rightarrow \chi_A\cap_B(u) = \chi_A(u)\chi_B(u)
\end{equation}

Характеристической функцией $U\setminus A$ будет $1 - \chi_A$. Мощностью множества будет равна:  

\begin{equation}
\sum_{u} \chi_A(u) = |A|
\end{equation}


Например, если $U = \{a,b,c,d,e,f,g\}$, $A = \{a,b,g\}$, $B = \{b,c,d\}$, то $\chi_A = [1, 1, 0, 0, 0, 0, 1]$, $\chi_B = [0, 1, 1, 1, 0, 0, 0]$, $\chi_U = [1, 1, 1, 1, 1, 1, 1]$, $\chi_A\cap_B(u) = \{b\} = [0, 1, 0, 0, 0, 0, 0]$, что равно $\chi_A(u)\chi_B(u) = [(1\ast 0), (1\ast 1), (0\ast 1), (0\ast 1), (0\ast 0), (0\ast 0), (1\ast 0)]$,  $U\setminus A = [(1-1), (1-1), (1-0), (1-0), (1-0), (1-0), (1-1)] = [0, 0, 1, 1, 1, 1, 0]$.

Объединение произвольного набора подмножеств $A_1 \cup A_2 \cup ... \cup A_n$ совпадает с дополнением пересечения дополнений этих подмножеств, что можно записать так: 

\begin{equation}
\chi_{A_1\cup...\cup A_n} = 1 - (1 - \chi_{A_1})...(1 - \chi_{A_n})
\end{equation}

Если раскрыть скобки в правой части тождества, мы получим следующее (что легко проверить, просто проведя умножение самостоятельно): 

\begin{equation}
\sum_{i}\chi_{A_i} - \sum_{i<j}\chi_{A_i}\chi_{A_j} + \sum_{i<j<k}\chi_{A_i}\chi_{A_j}\chi_{A_k} - ...
\end{equation}

Что и требовалось доказать. 

\section{Отображения множеств}
Этот раздел обязателен для освоения. 

Упорядоченной парой называется пара элементов некоторого множества, такая, что: $\forall a,b\in A \Rightarrow (a,b) \neq (b,a)$. Рассмотрим теперь два множества - $A$ и $B$. Тогда декартовым произведением $A\times B$ называется множество всех таких упорядоченных пар вида $(a,b)$, что $a \in A \wedge b \in B$. 

Всякое подмножество $C \subset A\times B$ называется отношением множеств $A$ и $B$. Если множества $A$ и $B$ совпадают, то говорят о бинарном отношении на множестве $A$, а декартово произведение называют декартовым квадратом и записывают вот так: $A^2$. Если отношение задано на конечном семействе (мощности n) множеств $A$, то такую операцию называют n-рной, а декартово произведение множеств $A$ записывают вот так: $A^n$. 

Например, если $A = \mathbb{R}$, то на множестве $B \subset\mathbb{R}^2$ отношением будет множество всех таких пар $(x,y)$, где $x,y \in \mathbb{R}$, что $x\neq y$. Это пример, понятно, что можно составить и другие отношения в $\mathbb{R}^2$. 


\textbf{Отображением (функцией)} $f:A_1 \rightarrow A_2$ называется такое отношение $B_f \subset A_1 \times A_2$ что для всякой пары элементов $a \in A_1 \wedge b \in A_2 \Rightarrow (a,b) \in B_f$, то при $f(a_1) = b \wedge f(a_2) = b \Rightarrow f(a_1) = f(a_2)$. То есть, для каждого $a \in A_1$ существует единственный (далее $\exists!$) $b \in A_2$. Существенно здесь также и то, что для $\forall a\in A_1$ $\exists b \in A_2$, такой, что $(a,b) \in B_f$, но это не всегда работает в обратную сторону: если $b \in A_2$, то это ещё не значит, что $\exists a$, такое, что $(a,b)\in B_f$. Это обусловлено тем, что определение функции $f$ не запрещает следующее: из $(a_1,b)$ и $(a_2,b)$ $\nRightarrow a_1 = a_2$, например, у уравнения вида $y=x^2$ в множестве $\mathbb R$ два корня, то есть, двум $x$ соответствует один $y$. Множество $B_f$ (всех таких пар элементов двух множеств) называется \textbf{графиком функции}. Значением функции $f$ на элементе $a \in A_1$ называется элемент $b \in A_2$.

Просто думайте про функцию, как про набор правил, который \textbf{каждому} элементу некоторого множества $A_i$ ставит в соответствие единственный элемент некоторого другого множества $A_j$. 

\textbf{Преобразование множества} - это отображение множества в себя. Тождественным преобразованием, обозначаемым $Id_A: A\rightarrow A$ называется такое преобразование множества $A$, которое $\forall a\in A \rightarrow a$, а $Id_A$ состоит из пар элементов множества $A$ вида $(a,a) \Leftrightarrow a=a$. График такого отображения - диагональ. 

\textbf{Композиция отображений} - это последовательное выполнение нескольких отображений. Пусть $f: A_1 \rightarrow A_2$ и $g: A_2 \rightarrow A_3$, тогда $g\circ f: A_1 \rightarrow A_3$. Эта функция ставит в соответствие всякому $a\in A_1$ некоторый $g(f(a))\in A_3$.
 
\begin{equation}
{\xymatrix{
	& {A_1\ar[rr]_{f} \ar[drr]_{g\circ f}} && {A_2\ar[d]_{g}} 
	\\
	&&&{A_3} 
		 }}
\end{equation}

В $f(x)=y$, $y$ - образ $x$, а $x$ - прообраз $y$. Образом всего множества $A_1$ при отображении $f(A_1)$, где $f:A_1 \rightarrow A_2$, называется множество всех таких $y \in A_2$, у которых существует прообраз в $A_1$. Прообразом подмножества $S_2 \subset A_2$ называется совокупность всех таких элементов множества $x \in A_1$, что образ каждого из них лежит в $S_2$ - $\forall x\in A_1 \Rightarrow f(x)\in S_2$.

\textbf{Виды отображений}: 
\begin{itemize}
	\item Инъективным отображением называется такое отображение $f:X\rightarrow Y$, что из $f(x) = f(y) \Leftrightarrow x=y$. Например, функция вида $y=x^2$ не инъективна, а вот функция $y=x$ -- инъективная функция. 
	\item Сюръективным отображением называется такое отображение $f:X\rightarrow Y$, что для $\forall y\in Y$ $\exists x\in X$ такой, что $f(x) = y$. То есть, у всякого элемента области значений функции $f$ $\exists$ прообраз в множестве $X$. Очевидно, что если отображение $f: X\rightarrow Y$ -- сюръекция, но не инъекция, то мощность $| X | > | Y |$.  
	\item Биективным отображением называется такое отображение, которое является сюръективным и инъективным одновременно.

\end{itemize}

\textbf{Задача}. Докажите, что, если $f: X\rightarrow Y$ биекция, то $\exists g:Y\rightarrow X$ такая, что $f \circ g$ -- тождественное преобразование $X$ (переводящее всякий элемент сам в себя), а $g \circ f$ -- тождественное преобразование $Y$.

\textbf{Определение}. Множество всех функция из множества $A$ в множество $B$ обозначается $ A^B $. Отсюда же следует, что $0^0 = 1$, то есть, множество функций из пустого множества в пустое множество одноэлементно (является синглетоном). 

\section{Понятие отношения на множестве}

\textbf{Определение}. Пусть у нас есть некоторое множество $X$ произвольной природы, тогда рассмотрим его декартово произведение на себя -- $X\times X = X^2$. Пусть $B$ -- произвольное подмножество $X^2$. Тогда скажем, что на множестве $X$ задано \textit{бинарное отношение} $B$.  То есть, все пары элементов из $X$, входящие в подмножество $B\subset X^2$, считаются удовлетворяющими некоторому бинарному отношению, например, отношению эквивалентности,  порядка, подобия, параллельности и так далее. Если элементы $x_1,x_2\in X$ находятся в отношении $B$, то есть пара $(x_1,x_2)\in B\subset X^2$, то это можно записать так: $x_1 B x_2$. 

Стоит заметить, что подмножество $X^2\subset X^2$, то есть, подмножество, совпадающее со всем множеством, -- тоже вариант некоторого отношения, определённого на множестве $X$. Просто в этом отношении состоят абсолютно все элементы исходного множества. Дополнение о котором стоит подумать самостоятельно: такое отношение стягивает исходное пространство в точку. 

\textbf{Определение}. Множество всех первых компонент пар элементов $X$, входящих в $B\subset X^2$, называется областью определения отношения $B$ и обозначается как $Dom B$. 

\begin{equation}
Dom B = \{x\in X\mid\exists y, (x,y)\in B\}
\end{equation}

\textbf{Определение}. Множество всех вторых компонент пар элементов $X$, входящих в $B\subset X^2$, называется областью значения отношения $B$ и обозначается $Im B$. 
\begin{equation}
Im B = \{y\in X\mid\exists x, (x,y)\in B\}
\end{equation}

В английской литературе обычно $Im$ заменяется на $Ran$ и читается как \textit{range} of a relation $R$ (сразу становится понятно, почему произвольное отношение (relation) обозначают как $R$).


\textbf{Определение}. $fld(R) = Im(R) \cup Dom(R)$, читается как \textit{field} of a relation $R$. 

Полезная интуиция: представьте себе некоторое множество точек плоскости $R\subset \mathbb{R}^2$. Каждая такая точка имеет две координаты, то есть, задаётся некоторой парой чисел $(x,y)$. Тогда это множество точек $R$ -- некоторое отношение на $\mathbb{R}\times \mathbb{R}$. Теперь представим себе наглядно $Im(R)$ и $Dom(R)$ как проекции нашего подмножества на координатные оси $Ox,Oy$, то есть, как множества точек вида $\{(x,0)\}$ и $\{(0,y)\}$, где $\forall x,y$ верно, что $\exists$ такие точки в $R$, что одна из их координат -- $x$ или $y$. 

\textbf{Определение}. $n$-арным отношением $B$, заданным на множествах $X_1,\dots, X_n$, называется подмножество декартового произведения этих множеств: $B\subset X_1 \times,\dots,\times X_n$. То что $n$-ка элементов находится в отношении $B$ обозначается как $B(x_1,\dots,x_n)$. 

\textbf{Определение}. Универсальным отношением $B$ называют такое подмножество $B\subset X_1 \times,\dots,\times X_n$, что $B = X_1 \times,\dots ,\times X_n$. 

\textbf{Определение}. Пустым отношением $B$ называют пустое подмножество $B = \emptyset\subset X_1 \times , \dots , \times X_n$. 



\subsection{Свойства отношения}

Бинарное отношение $B$ на множестве $X$ может обладать различными свойствами: 

\begin{itemize}
	\item рефлексивность: для $\forall x\in X$ верно, что $xBx = xBx$. Бинарное отношение $B$ является рефлексивным, тогда и только тогда, когда тождественное отношение $Id_X$ является его подмножеством $Id_X\subset B$. Где $Id_X = \{(x,x)\mid x\in X\}$; 
	\item антирефлексивность: для $\forall x\in X$ верно, что $\neg xBx$. То есть, ни один элемент множества $X$ не находится в отношении $B$ с самим собой;
	\item корефлексивность: для $\forall x,y\in X$ верно, что $xBy \Rightarrow x = y$;
	\item симметричность: для $\forall x,y\in X$ верно, что $xBy \Rightarrow yBx$;
	\item антисимметричность: для $\forall x,y\in X$ верно, что $xBy \wedge yBx \Rightarrow x = y$;
	\item асимметричность: для $\forall x,y\in X$ верно, что $xBy \Rightarrow \neg(yBx)$. Где $\neg$ -- отрицание, читается как \textit{негация};
	\item транзитивность: для $\forall x,y,z\in X$ верно, что $xBy \wedge yBz \Rightarrow xBz$; 
	\item евклидовость: для $\forall x,y,z\in X$ верно, что $xBy \wedge xBz \Rightarrow yRz$. Название отношение получило по аналогии с первой аксиомой из «Начал» Евклида: "равные одному и тому же равны и между собой". Очевидно, что евклидово отношение не обязательно транзитивно, а транзитивное евклидово. Рефлексивное и симметричное отношение является евклидовым, тогда и только тогда, когда оно транзитивно;
	\item полнота: для $\forall x,y\in X$ верно, что $xBy \vee yBx$;
	\item связность: $\forall x,y\in X$ верно, что $x\neq y \Rightarrow xBy \vee yBx$;
	\item connex: для $\forall x,y\in X$ верно, что $xBy \vee yBx\vee x\neq y$;
	\item трихотомия: для $\forall x,y\in X$ верно, что $xBy \veebar yBx \veebar x\neq y $. Где $\veebar$ -- строгая дизъюнкция. 
	
\end{itemize}

\subsection{Виды отношений}
Некоторые из этих отношений будут разобранны чуть подробнее ниже. 

\begin{itemize}
	\item Рефлексивное симметричное транзитивное отношение называется отношением эквивалентности;
	\item Рефлексивное антисимметричное транзитивное отношение называется отношением (частичного) порядка;
	\item Антирефлексивное антисимметричное транзитивное отношение называется отношением строгого порядка;
	\item Полное антисимметричное (для любых $\forall x,y\in X$ выполняется $ xBy \vee  yBx$) транзитивное отношение называется отношением линейного порядка;
	\item Рефлексивное транзитивное отношение называется отношением квазипорядка.
\end{itemize}

\textbf{Замечание}. Очевидно, что функция от одной переменной тоже является бинарным отношением $B$, определенным на некотором множестве $X$. Всякому значению $x$ в отношении $xBy$ соотвествует единственное значение $y$. Также можно заметить, что это отношение, заданное на всех элементах множества $X$, то есть, всякий элемент множества $X$ находится в отношении $B$ с  $y$. В случае \textit{биекции} можно говорить о таком отношении $B$ на $X$, что всякое $x$ находится в отношении $B$ с единственным $y$ и наоборот, всякое $y$ находится в отношении $B$ с единственным $x$. 

\section{Отношение эквивалентности}
Этот раздел обязателен для освоения. 

\textbf{Отношением эквивалентности} на множестве $X$ называется такое подмножество $X \times X$ для элементов которого выполняются следующие аксиомы:
\begin{itemize}
	\item \textbf{Рефлексивость}: для $\forall x\in X: x\sim x$
	\item \textbf{Коммутативность}: для $\forall x\in X: x\sim y \Leftrightarrow y\sim x$
	\item \textbf{Транзитивность}: для $\forall x,y,z\in X: x\sim y$ и $y\sim z \Rightarrow x\sim z$
\end{itemize}	
Если эти аксиомы выполнены, то такое отношение называется отношением эквивалентности на множестве $X$.

\textbf{Пример}. Пусть $X = \{1,2,3,4,5,6\}$, а $B = $

\textbf{Классом эквивалентности} элемента $x\in X$ называется "множество" всех таких $y\in X$, что $x\sim y$. В классической теории множеств (аксиоматической) "класс эквивалентности" не является множеством в прямом смысле этого слова. Представителем класса эквивалентности называется всякий элемент множества $X$, который лежит в нем. То есть, задавая отношение эквивалентности на некотором множестве, мы "разбиваем" его на классы эквивалентности. Может быть так, что, например, у нас на множестве всего один класс эквивалентности -- все элементы множества попарно эквивалентны. Может быть и так, что классов столько же, сколько и элементов в самом множестве, то есть, всякий элемент эквивалентен только сам себе. 

Классы эквивалентности не пересекаются, что очевидно. Если элемент лежит сразу в нескольких классах, то эти классы совпадают.

Например, равномощность двух подмножеств некоторого множества -- отношение эквивалентности. Ещё пример: пусть $f$ -- некоторая функция из множества $X$ в $Y$. Всевозможные прообразы $y\in Y$ образуют естественное разбиение множества $X$ на классы. Это верно и в обратную сторону.

\textbf{Пример}. Парламент, члены парламента и представляемые ими партии в народе. 

\textbf{Определение}. Фактор-множество некоторого множества -- множество всех классов эквивалентности множества $X$. Обозначается это так: $X\diagup\sim$. Разбиение же множества $X$ на классы эквивалентности -- \textit{факторизация}. 

\textbf{Пример}. Разбиение на классы, с "вытягиванием" по транзитивности элементов из $X$ 


$\forall x\in X$ однозначно определён некоторый класс в $X\diagup\sim$. То есть, существует сюръективное отображение из $X$ в $X\diagup\sim$. 

Если множество снабжено структурой, то часто отображение $X\rightarrow X\diagup\sim$  можно использовать чтобы снабдить фактормножество  $X\diagup\sim$ той же структурой, например топологией. В этом случае множество $X\diagup\sim$ с индуцированной структурой называется \textbf{факторпространством}.

\textbf{Пример}. Фактормножество на множестве $\mathbb{Z}$, где отношение эквивалентности задается разбиением множества $\mathbb{Z}$ на три класса эквивалентности: $\mathbb{Z}_0, \mathbb{Z}_1, \mathbb{Z}_2$. Каждый класс содержит такие целые числа, которые при делении на $3$ дают один остаток. 

\textbf{Определение}. Отображение проекции 

\section{Семейства и покрытия}
При первом прочтении этот раздел можно пропустить. 

Пусть $X$ -- некоторое множество, тогда семейством его подмножеств $\sigma = \{X\}$ называется некоторое множество, элементами которого являются подмножества множества $X$. Объедиение всех тех подмножеств в $X$, которые входят в $\sigma$, как элементы, назовём телом семейства $\sigma$ и обозначим $\varsigma$. То есть, $\varsigma = \bigcup_i X_i$ где $\forall X_i \subseteq X$ и $\bigcup_i X_i\subseteq X$. Обозначим $\sigma_j$ подсемейство семейства $\sigma$, которое состоит из всех тех элементов семейства, которые в множестве $X$ имеют не пустое пересечение с подмножеством $X_j$. 

Семейство $\sigma$ называется покрытием подмножества $X_i \subseteq X$, если $X_i \subseteq\varsigma$. Это верно и для всего множества $X$, то есть, семейство $\sigma$ называется покрытием множества $X$, если оно совпадает с телом семейства $\sigma$, то есть $\varsigma = X$, в этом случае подпокрытием покрытия множества $X$ будет называться всякое подсемейство семейства $\sigma$. 

Кратностью семейства $\sigma$ в точке $x\in X$ является множество всех элементов семейства $\sigma$ (читай, подмножеств $X$), которые содержат $x$. Семейство называется конечнократным, если кратность семейства $\sigma$ во всякой точке $x\in X$ конечна. Существуют счетнократные семейства и несчетнократные семейства. 

Вписанным покрытием $\alpha$ в покрытие $\beta$ множества $X$ (общего для обоих покрытий) называется такое покрытие, что всякий элемент покрытия $\alpha$ является подмножеством хотя бы одного элемента покрытия $\beta$. В частности, покрытие $\alpha$ вписано в покрытие $\beta$, если $\alpha$ является подпокрытием $\beta$. 

\section{Отношение порядка}
Этот раздел обязателен для освоения. 

\textbf{Отношением частичного порядка} на множестве $X$ называется такое отношение $\preceq$, для которого выполняются следующие аксиомы: 
\begin{itemize}
	\item \textbf{Рефлексивность}: для $\forall x\in X: x\preceq x$ и $x\succeq x$.
	\item \textbf{Транзитивность}: $x\preceq y$ и $y\preceq z$, то $x\preceq z$.
	\item \textbf{Асимметричность}: $x\preceq y$ и $y\preceq x$, то $x=y$.
\end{itemize}
Если же отношение порядка определено для любых двух элементов множества, то такое отношение называется \textbf{отношением линейного порядка}. Например, множество $\mathbb R$ линейно упорядочено, а вот множество $\mathbb C$ нет. 

Как следует из определения, всякое линейно упорядоченное множество является и частично упорядоченным, но в обратную сторону это не верно. Важным примером частично упорядоченного множества является множество подмножеств некоторого множества $X$, упорядоченное по включению. 

\begin{equation}
M_i \prec M_j\Leftrightarrow M_i\subset M_j \subseteq X
\end{equation}

\textbf{Определение}. Если в некотором упорядоченном множестве имеется такое отношение: $x_1 \prec x_2 \prec x_3$, то элемент $x_2$ лежит между элементами $x_1$ и $x_3$. Множество всех таких $x_2$, которые лежат между $x_1$ и $x_3$ в упорядоченном множестве $X$, называется \textbf{интервалом} $(x_1;x_3)$ упорядоченного множества $X$. 

Существуют и пустые интервалы (не содержащие элементов множества) в упорядоченном множестве, например, интервалы вида $(n;n+1)$ в $\mathbb{N}$.

\textbf{Определение}. Максимальным элементом упорядоченного множества $X$ называется такое $x_i\in X$, что $\forall x_j\in X$ верно $x_j\preceq x_i$. Минимальный элемент определяется аналогично. 
В частично упорядоченном множестве может быть много минимальных и максимальных элементов, в линейно упорядоченном множестве они единственны. 
Интервал, которому принадлежат максимальный и минимальный элементы, называется сегментом $[x_1,x_3]$. 

\textbf{Определение}. Изоморфизмом (в теоретико множественном смысле) называется биекция двух множеств, сохраняющая порядок прообраза на образе. То есть, это такое $f: X\rightarrow Y$, что для $\forall (x,y)\in X\times X$, для которых выполняется $x\preceq y$, верно, что $f(x,y) = (f(x)\preceq f(y))$. Обозначается это $X\cong Y$. 
\\
\textbf{Теорема}. Равномощные конечные линейно упорядоченные множества изоморфны. 
\\
Это следует из того, что такое множество изоморфно естественно упорядоченному подмножеству $\mathbb{N}$, а именно -- $\{1,2,...,n\}$.
\\
Примеры равномощных, но не изоморфных линейно упорядоченных множеств:

\begin{itemize}
	\item Отрезок $[0,1] \ncong \mathbb{R}$. Потому что в первом множестве есть наименьший и наибольший элементы, а в $\mathbb{R}$ их, очевидно, нет. При изоморфизме наибольшие элементы переходят в наибольшие, а наименьшие в наименьшие. 
	\item $\mathbb{Z}\ncong\mathbb{Q}$. Допустим, что $f:\mathbb{Z}\rightarrow\mathbb{Q}$ изоморфизм. Рассмотрим два последовательно идущих друг за другом элемента $\mathbb{Z}$, например, $2$ и $3$. При изоморфизме им должны соответствовать два рациональных числа $f(2)\prec f(3)$, но тогда рациональным числам между $f(2)$ и $f(3)$ должны соответствовать некие целые числа, которых нет. 
\end{itemize}

\textbf{Теорема}. Любые два счетных плотных линейно упорядоченных и неограниченных (ни снизу, ни сверху) можества -- изоморфные множества. 


\textbf{Теорема}. Всякое счетное линейно упорядоченное множество изоморфно некоторому подмножеству множества $\mathbb{Q}$. Или любому другому изоморфному $\mathbb{Q}$ множеству. 


\textbf{Определение}. Изоморфизм множества в себя называют автоморфизмом. Например тождественное отображение, когда всякий элемент частично упорядоченного множества отображается сам в себя.


\textbf{Упражнение}. Покажите, что не существует автоморфизма упорядоченного множества $\mathbb{N}$, отличного от тождественного. 

 

Два изоморфных конечных множества также являются эквивалентными, то есть равномощными. Поэтому можно построить понятие порядкового типа, как и класса эквивалентности: такого множества, в котором элементами всякого типа являются попарно изоморфные множества.  Мощностью типа будет называться мощность всякого элемента этого порядкового типа. Порядковые типы конечных множеств могут быть биективно отображены в множество мощностей конечных множеств. Заметьте, это верно только для конечных множеств. 

\textbf{Определение}. Порядково выпуклым подмножеством $X_i$ упорядоченного множества $X$, называется такое множество, если вместе со всякими двумя элементами $x_i\preceq x_j$ множество содержит и ограниченный ими сегмент $[x_i;x_j]$. 

\textbf{Определение}. Упорядоченной суммой множеств назовём такое объединение двух непересекающихся (дизъюнктных) частично упорядоченных множеств $X$ и $Y$, что порядок на их объединении $X\cup Y$ задан следующим образом: для всех элементов $X$ и $Y$ сохраняется тот частичный порядок, который был на них задан, и любой элемент $X$ считается меньше любого элемента множества $Y$. Обозначим это $X\uplus Y$. Очевидно, что $X\uplus Y \neq Y\uplus X$. 

Пусть $(X;\preceq_{X})$ и $(Y;\preceq_{Y})$ два упорядоченных множества. Порядок на их произведении $X\times Y$ можно определить несколькими способами. Можно считать, что $(x_1,y_1)\preceq(x_2,y_2)$, тогда и только тогда, когда $x_1\preceq x_2$ и $y_1\preceq y_2$. Такой порядок будет частичным, даже тогда, когда $X$ и $Y$ упорядочены линейно. Дабы получить линейный порядок, необходимо выбрать какой-то элемент пары, как "главный". Например, пусть это будет $y_i$. Тогда $(x_1,y_1)\prec (x_2,y_2)$, тогда, когда $y_1\prec y_2$. Если $y_1 = y_2$, то смотрим уже по $x$. Равенство же $(x_1,y_1) = (x_2,y_2)$ выполняется тогда и только тогда, когда $x_1 = x_2$ и $y_1 = y_2$. Очевидно, что подобное можно обобщить на произвольную по длине (конечную) последовательность декартовых произведений множеств. 


\textbf{Упражнение}. Сколько существует линейных порядков на множестве из $n$ элементов. 


\textbf{Упражнение}. Докажите, что всякий частичный порядок, заданный на конечном множестве, можно продолжить до линейного порядка (продолжить -- сохранить "старый" порядок, дополнив его до линейного). 
\section{Кардиналы и теорема Кантора-Бернштейна-Шредера}
Два множества называются равномощными, если между $A$ и $B$ существует биективное соответствие. Счетное множество равномощно $\mathbb{N}$. Если множество $A$ равномощно подмножеству $B$, то говорится, что мощность $A$ \textbf{не больше} мощности $B$. 

\textbf{Теорема}. Пусть $A$ и $B$ такие множества, что $A \subset B$ и $B \subset A$, то $A$ и $B$ равномощны. 
\textbf{Доказательство}. Пусть $f: A\hookrightarrow B$ -- вложение из $A$ в $B$, а $g: B\hookrightarrow A$ -- вложение из $B$ в $A$. Рассмотрим отображения $f \circ g: A \rightarrow A$ и $g\circ f: B\rightarrow A$. Обозначим за $A_0 = A\setminus g(B)$, за $B_0 = B\setminus f(A)$, и за $A_1$ множество $g(B_0)$. Определим индуктивно $A_{i+2}$, как $A_{i+2} = f \circ g(A_i)$. Определим $A_{\propto}$ как пересечение образов отображений $(f\cdot g)^i$ для $\forall i$ (под $(f\cdot g)^i$ предполагается композиция $f\cdot g$ с собой $i$ раз). Таким образом, $A$ разбивается в дизъюнктное объединение подмножеств $A = A_0 \sqcup A_1\sqcup ... \sqcup A_{\propto}$. Применим аналогичную процедуру к $B$, получив разбиение $B = B_0 \sqcup B_1 \sqcup ... \sqcup B_{\propto}$. 

Видно, что $f$ -- биекция из $A_{\propto}$ в $B_{\propto}$: любой элемент из  $A_{\propto}$ является прообразом элемента из $B_{\propto}$ и наоборот. $f$ задаёт биекцию из $A_0 \sqcup A_1\sqcup ...$ в $B_0 \sqcup B_1 \sqcup ...$, а $g$ из $B_0 \sqcup B_1 \sqcup ...$ в $A_0 \sqcup A_1\sqcup ...$. Получилось разбиение $A$ и $B$ на непересекающиеся подмножества, которые попарно биективны. $\qed$

\section{Индукция и базовая логика}

\textbf{Определение}. Пусть $A(n)$ некоторое свойство натурального числа $n$. Если нам удалось доказать, что $A(n)$ верно для $n$, исходя из предположения, что $A(m)$ верно для всякого $m$ меньшего $n$, то свойство $A(n)$ верно для всех натуральных чисел. 

\textbf{Теорема}. Следующие свойства частично упорядоченного множества равносильны:
\begin{itemize}
	\item Любое непустое подмножество $X$ имеет минимальный элемент.
	\item Не существует бесконечной строго убывающей последовательности элементов множества $X$: $x_0>x_1>x_2>...$. 
	\item Для множества $X$ верен принцип индукции в следующей форме: если для $\forall x\in X$ из истинности $A(y)$ для $\forall y<x$ следует истинность $A(x)$, то свойство $A(x)$ верни при всех $x$. 
	\begin{equation}
		\forall x(\forall y((y<x)\Rightarrow A(y)) \Rightarrow A(x))\Rightarrow\forall x A(x)
	\end{equation}
\end{itemize}

\textbf{Доказательство}. Начнём с доказательства эквивалентности первых двух пунктов, а для этого докажем, что из первого пункта следует второй и наоборот. 
Если существует бесконечная строго убывающая последовательность (отрицание второго пункта), то очевидно, что множество её значений не имеет наименьшего элемента. Поэтому, если таких последовательностей в нашем множестве $X$ нет, то и подмножеств без наименьшего элемента тоже нет. То есть, нет подмножеств, которые были бы неограничены снизу. 
В обратную сторону. Если множество $X$ --- непустое множество, не имеющее минимального элемента, то можно построить бесконечную убывающую последовательность. Сделаем это так: возьмем произвольный элемент множества, по определению он будет не минимальным, а значит, существует элемент меньший его. И так далее. Получается бесонечная строго убывающая последовательность. 
Выведем третий пункт из первого. Пусть $A(x)$ --- произвольное свойство элементов множества $X$, присущее не всем элементам множества. Рассмотрим множество тех элементов $X$, для которых свойство $A$ неверно, обозначим его $Y\subset X$. Пусть $y$ --- минимальный элемент $Y$. Тогда всякий меньший элемент, лежащий в $X\setminus Y$, такой что $x<y$, имеет верным для себя свойство $A$. Но тогда по предположению оно должно быть выполнено и для элемента $y$. 
В обратную сторону, выведем первый пункт из третьего. Пусть $Y$ --- подмножество множества $X$ без минимальных элементов. Докажем по индукции, что $Y$ пусто. Возьмем за $A(x)$ свойство $x\notin Y$. В самом деле, если $A(x)$ верно для всех $x<y$, то никакой элемент меньший $y$ не лежал бы в $Y$. Тогда, если $y\in Y$, то он является наименьшим --- противоречие. 

\textbf{Определение}. Множества, обладающие этими тремя свойствами (любым из трех), называются фундированными. Примером такого множества может служить множество $\mathbb{N}$. 

\textbf{Замечание}. Всякое частично упорядоченное множество, такое, что любое его подмножество имеет наименьший элемент, является линейно упорядоченным. Это следствие из того, что всякое двуэлементное подмножество частично упорядоченного множества имеет наименьший элемент, а значит, всякие два элемента этого частично упорядоченного множества сравнимы. 

В математике что-то признаётся справедливым или истинным тогда и только тогда, когда это логически выведено из некоторых базовых, непротиворечивых предпосылок-принципов. В обычной же науке всякое утверждение, признаваемое истинным, выводится не дедуктивно (из общих принципов), но индуктивно. Однако эмпирическая индукция и математическая индукция --- разные явления. Эмпирическая индукция не обладает правом на полноту своих выводов, математическая индукция наоборот. Всякий эмпирическй закон или объективный факт --- то, что может быть уточнено или видоизменено в будущем, то есть, имеется ненулевая вероятность того, что утверждение окажется ложным. 


\section{Вполне упорядоченные множества}

\textbf{Определение}. Вполне упорядоченным множеством называется фундированное линейно упорядоченное множество. В нем всякое непустое подмножество имеет наименьший элемент.
 
\textbf{Примеры}. $\mathbb{N}$ или $\mathbb{N}^k$.

\textbf{Свойства вполне упорядоченных множеств}:

\begin{itemize}
	\item Вполне упорядоченное множество имеет наименьший элемент.
	\item У всякого элемента $x$ вполне упорядоченного множества существует элемент $y$, непосредственно следующий за ним. То есть, такой $y>x$, что не существует $z$ такого, что $y>z>x$. Если множество всех больших $x$ элементов непусто, то в нём есть наименьший. 
	\item Некоторые элементы вполне упорядочен ного множества могут не иметь явного предыдущего элемента. Например, наименьший элемент. Такие элементы называются \textit{предельными}. 
	\item Всякий элемент вполне упорядоченного множества представим в виде $x_m + n$, где $x_m$ --- предельный, а $n\in\mathbb{N}$. 
	\item Всякое ограниченное сверху подмножество вполне упорядоченного множества, имеет точную верхнюю грань. Подмножество $Y\subset X$ вполне упорядоченного множества $X$ называется ограниченным сверху, если оно имеет верхнюю границу, то есть такой $x\in X$, что для $\forall y\in Y$ верно, что $y\leq x$. Если среди всех верхних границ данного подмножества $Y$ есть наименьшая, то она называется точной верхней гранью. А так как множество верхних границ ограниченного сверху подмножества вполне упорядоченного множества непусто, то оно имеет наименьший элемент --- точную верхнюю границу рассматриваемого множества. 
	
\end{itemize}

\section{Аксиоматическое построение числовых множеств}

Определение упорядоченного поля, индуктивного множества. Теорема Фробениуса. 

\chapter{Немного теории категорий}

\section{Общие положения и мотивировка изучения}

\section{Коммутативные диаграммы}




\chapter{Общая алгебра}
\section{Основные понятия}
Предмет алгебры -- изучение алгебраических структур. То есть, таких множеств, на которых определены некоторые наборы операций и отношений. 
Под операцией в множестве $X$ понимается отображение $f: X\times X \rightarrow X$, то есть некоторое правило, по которому двум любым элементам множества $X$ ставится в соответствие другой элемент этого множества.  

Под $n$-арной операцией на $X$ понимается отображение декартового произведения $n$ экземпляров множества с себя $f:X^n\rightarrow X$. А нульарная операция, по определению, просто выделенный элемент множества $X$. 

\textbf{Определение}. Пусть на множестве $X$ задана какая-то алгебраическая структура, то есть, операция, удовлетворяющая некоторому набору требований (например, коммутативности, ассоциативности и так далее), тогда запись $(X,*,e)$ называется сигнатурой этой структуры. Всё то, что внесено в сигнатуру, должно сохраняться при гомоморфизме структур. 

\begin{equation}
X\times X \rightarrow X
\end{equation}

Т.е. это правило, которое каждой паре элементов $(x_i;x_j)$ множества $X$ ставит в соответствие какой-то элемент $x_f\in X$. 

\textbf{Определение}. Моноидом называется такое множество $X$, композиция бинарных операций на котором удовлетворяет закону ассоциативности, а также существует единичный элемент. Так что, в частности, $X$ не пусто.

Пусть $X$ — моноид, а $x_1, \dots, x_n$ его элементы, где $i \in \mathbb{Z}$ и $n > 1$. Определим их произведение по индукции 

\begin{equation}
	\prod\limits_{i=1}^n x_i = x_1\dots x_n = (x_1\dots x_{n-1})x_n
\end{equation}

Верно правило: 

\begin{equation}
	\prod\limits_{i=1}^{n} x_i \cdot \prod\limits_{j=1}^{m} x_{n+j} = \prod\limits_{j=1}^{m+n} x_j
\end{equation}

Что, по сути, является следствием свойства ассоциативности. 

Пустое произведение равно единичному элементу $\prod\limits_{i=1}^0 = e$.

\textbf{Определение}. Если операция, определённая на моноиде $X$, коммутативная, то моноид $X$ называется коммутативным моноидом. 


\textbf{Утверждение}. Если $X$ — коммутативный моноид, $x_1,\dots,x_n$ его элементы, а $\psi$ — биективное отображение множества индексов $(1,\dots,n)$ на себя, то:

\begin{equation}
	\prod\limits_{i=1}^n x_{\psi(i)} = \prod\limits_{i=1}^n x_{i}
\end{equation}

Иными словами, порядок применения операции не важен. 

Если $x\in X$, где $X$ — моноид, то можно определить степень элемента $x^n$ как $\prod\limits{1}^n x = x^n$. Откуда явно следует, что $x^0 = e$. 

Также очевидно, что в коммутативном моноиде будет выполняться следующее: $(xy)^n = x^n\cdot y^n$ и $x^{n+m} = x^n\cdot x^m$.

Можно определить следующую конструкцию: если $X_i \subset X$ и $i>1$, то возможно рассмотреть произведение подмножеств моноида $X$ — $X_i X_j$, которое будет состоять из всех произведений вида $x_i x_j$, где $x_i \in X_i$ и $x_j \in X_j$. По индукции определяется произведение произвольного числа подмножеств моноида. 
Очевидно, что $XX = X$, так как моноид замкнут относительно своей бинарной операции, то есть, содержит все попарные произведения своих элементов. 

\textbf{Пример}. Дабы показать разнообразие проявлений таких алгебраических структур в математике, построим моноид на подмножествах топологического пространства. Если вы не знаете, что это такое, то не расстраивайтесь – пример играет чисто ознакомительную роль. Допустим, что $X$ — компактное и связное топологическое пространство. Рассмотрим множество классов $\{\sigma_i \}$ гомеоморфных подпространств $X$ – классов эквивалентности подпространств, где эквивалентными признаются гомеоморфные пространства. Определим бинарную операцию на классах подпространств следующим образом: пусть $X_i$ и $X_j$ — компактные подпространства изначального пространства $X$, $D_i \subset X_i$ — диск в подпространстве $X_i$, а $D_j \subset X_j$ — диск в подпространстве $X_j$. Пусть $C_i$ и $C_j$ — границы дисков $D_i$ и $D_j$. Пусть $D_i` = D_i \setminus C_i$ и $D_j' = D_j \setminus C_j$ — внутренности дисков. Приклеим $X_i \setminus D_i`$ к $X_j \setminus D_j'$ отождествив $C_i$ с $C_j$. 
Тогда класс подпространств, гомеоморфных двумерной сфере $S^2$, будет соответствовать нейтральному элементу в нашем моноиде — вырезав диск без границы из сферы мы получим двумерный диск с границей, которым можно идеально заклеить вырезанный из другого подпространства диск без границы, что, в общем, очевидно. Таким образом, мы получаем структуру моноида на множестве классов $\{\sigma_i\}$.  

Кроме того, если $\tau$ обозначает класс эквивалентности тора, а $\pi$ — класс проективной плоскости, то всякий элемент из $\{\sigma_i\}$ представляется в форме: 

\begin{equation}
	\sigma_i = n\tau + m\pi,
\end{equation}

где $n\in \mathbb{Z}$ и $n\geq 0$, а $m =0$, $1$ или $2$. Также справедливо равенство $3\pi = \tau + \pi$. 


\textbf{Определение}. Пусть даны два множества $N$ и $M$ с сигнатурами $(N;\cdot)$ и $(M;\circ)$. Тогда изоморфизмом \textit{алгебрачиских структур} $(N;\cdot)$ и $(M;\circ)$ называется такая биекция $f:M\rightarrow N$, что операция, заданная для прообразов, сохраняется для образов. 


\begin{equation}
	f(m_1\cdot m_2) = f(m_1)\circ f(m_2)
\end{equation}

Что верно для $\forall m\in M$. Это обозначается $(N;\cdot)\simeq(M;\circ)$, а структуры называются изоморфными.

\textbf{Определение}. Пусть задана алгебраическая структура на $X$ -- $(X;\cdot;e)$, такая, что $X$ -- моноид. Тогда элемент $x\in X$ называется регулярным слева, если на него можно сокращать слева:

\begin{equation}
\forall y,z\in X: x\cdot z = x\cdot y \Rightarrow z = y
\end{equation} 

Аналогично определяется \textit{регулярный справа} элемент (на него можно сокращать справа). 

\textbf{Замечание}. Любой обратимый слева/справа элемент регулярен слева/справа, но не наоборот. Это выполняется в любой ассоциативной структуре. 

\textbf{Доказательство}. Пусть $u\cdot x = e$, $x\cdot y = x\cdot z\Rightarrow (u\cdot x)\cdot y = u \cdot (x\cdot y) = u\cdot (x\cdot z) = (u\cdot x) \cdot z \Rightarrow y = z$. Видно, что в этом доказательстве активно используется свойство ассоциативности. 


Моноиды возникают в различных областях математики; например, моноиды можно рассматривать как категории из одного объекта. Таким образом, моноиды обобщают свойства композиции функций. Также моноиды используются в информатике и в теории формальных языков.

\textbf{Пример}. Регулярного, но необратимого элемента. 

\textbf{Определение}. Группой называется такая алгебраическая структура $(G;mult; inv; e)$, для которой выполнены следующие аксиомы:
\begin{itemize}
	\item $\forall g\in G (g_1\cdot g_2)\cdot g_3 = g_1\cdot(g_2\cdot g_3) $ -- ассоциативность операций;
	\item $mult: G\times G\rightarrow G$, такая, что $\forall g_1,g_2\in G: g_1\cdot g_2\in G $ -- замкнутость относительно бинарной операции $mult$. 
	\item $inv: G\rightarrow G$, такое, что $\forall g\in G$ $\exists!$ $g^{-1}: g\cdot g^{-1} = e $ -- существование единственного обратного элемента для любого элемента группы (правый и левый обратные элементы совпадают). Это унарная операция, определенная в структуре благодаря единственности обратного, что возможно из-за ассоциативности бинарной операции. 
	\item $\exists e\in G$ такое, что $e\cdot g = g $ -- существование единственного нейтрального элемента (единственность -- свойство, которое выводится элементарно из аксиом группы). Это нульарная операция. 

\end{itemize}

По сути, группа – коммутативный моноид, в котором все элементы обратимы. 

\vspace{\baselineskip}

\textbf{Элементарные свойства групповой структуры}: 
\begin{itemize}
	\item Сокращение. $\forall x,y,z\in G$ верно, что $xy = xz \Rightarrow y = z$ и $yx = zx \Rightarrow y = z$; 
	\item Деление. $\forall h,g$ $\exists ! x$ такой, что $hx = g$ ($x=h^{-1}g$) или $xh = g$ ($x=gh^{-1}$). Это, очевидно, не одно и то же, если умножение некоммутативно. 
\end{itemize}

Если добавляется следующий пункт, то группа называется абелевой: 
\begin{itemize}
	\item $ \forall m\in M \Rightarrow m_1\cdot m_2 = m_2\cdot m_1 $ -- коммутативность
\end{itemize}

\textbf{Определение}. Подгруппой группы $M$ называется такое подмножество $M_1\subseteq M$, на котором определена та же операция, что и на $M$, а также выполнены все аксиомы группы $(M;\cdot)$. 


\textbf{Пример}. Структуру аддитивиной группы можно построить на множестве $\mathbb{Z}$, а подмножество $X\subset \mathbb{Z}$ $\forall z\in X: z\equiv 0 (mod 2)$, является подгруппой группы $\mathbb{Z}$. 

\textbf{Определение}. Порядком группы $G$ называется мощность носителя структуры группы. Для конечных групп -- число элементов группы. 

\textbf{Пример}. Диэдральной группой называется конечная группа, состоящая из $2n$ элементов, обозначается она $D_{2n}$. Структура группы базируется на представлениях о симметричных преобразованиях правильного $n$-угольника: $n$ поворотов и $n$ осевых симметрий.  Если $n$ нечетно, каждая ось симметрии проходит через середину одной из сторон и противоположную вершину. Если $n$ четно, имеется $\dfrac{n}{2}$ осей симметрии, соединяющих середины противоположных сторон и $\dfrac{n}{2}$ осей, соединяющих противоположные вершины. В любом случае, имеется $n$ осей симметрии и $2n$ элементов в группе симметрий. Отражение относительно одной оси, а затем относительно другой, приводит к вращению на удвоенный угол между осями. Очевидно, что все диэдральные группы, начиная с группы $D_3$ (включая группу $D_3$), -- не абелевы группы. 

В качестве упражнения проверьте выполнимость аксиом группы, а также постройте таблицу умножения (называемую ещё таблицей Кэли) для произвольного $n$-угольника. 


\textbf{Пример}. Симметрической группой $S_n$ (степени $n$) называется конечная группа, состоящая из множества биекций множества $X \rightarrow X$ в себя, относительно композиции биекций, где $n$ -- число элементов в множестве $X$. Или, что аналогично, множество перестановок $n$-элементного множества $X$. 

Единицей группы $S_n$ является тождественное отображение $Id_X : x\rightarrow x$. $Id_x \circ f = f = f \circ Id_X$. Порядок группы $S_n = n!$. Композиция отображений, очевидно, ассоциативна: $(f\circ g) \circ h = f\circ (g\circ h)$. Так как отображение биективно, то оно двусторонне обратимо, относительно композиции, то есть, для всякой такой функции $f$ существует единственная обратная функция $f^{-1} $. Иными словами, если $f$ -- биекция, то $\Leftrightarrow$, что $f$ обратима: $y = f(x)\Leftrightarrow x =f^{-1}(y)$. 

Группа $S_3$ изоморфна группе $D_6$. 

\subsection{Гомоморфизм}

\textbf{Определение}. Гомоморфизм -- морфизм в категории алгебраических систем, который является таким отображением из одной алгебраической системы в другую, что сохраняются основные операции и основные соотношения, то есть, структура алгебраической системы остается \textit{неизменной}. Отображение $f: X\rightarrow Y$ называется гомоморфизмом, если операции, например, следующего вида $(X,\cdot)$ и $(Y,\oplus)$ сохраняются: $f(x_1\cdot x_2) = f(x_1)\oplus f(x_2)$. 


\textbf{Определение}. Если $f: X\rightarrow Y$ -- гомоморфизм, то $Ker(f) = \{x\in X\mid f(x) = 0 \}$ -- ядро гомоморфизма $f$. 


\textbf{Пример}. Гомоморфизм $G_1 = (\mathbb{Z},+)$ и $G_2 = (\{-1,1\},\cdot)$, где четные переводятся в $1$, а нечетеные в $-1$. 

\textbf{Пример}. Пусть $ln: \mathbb{R_+}\rightarrow \mathbb{R}$, где $x \rightarrow ln(x)$. Тогда $ln(xy)=ln(x)+ln(y)$ -- гомоморфизм. 


Есть и более простые алгебраические структуры (приведём некоторые примеры): 
\begin{itemize}
	\item $\Omega$-Алгебра -- множество элементов структуры (носитель) и множество разрешенных операций (сигнатура)
	\item Группоид -- $\Omega$-Алгебра с одной бинарной операцией вида $M\times M\rightarrow M$
	\item Правая квазигруппа -- группоид, в котором для $\forall m_1, m_2\in M$ $\exists! x: x\cdot m_1 = m_2$ имеет единственное решение. Определено деление справа. 
	\item Левая квазигруппа -- группоид, в котором для $\forall m_1, m_2\in M$ $\exists! x: m_1\cdot x = m_2$ имеет единственное решение. Определено деление слева. 
	\item Квазигруппа -- левая и правая квазигруппы вместе (не ассоциативна, в отличие от полугруппы и группы)
	\item Лупа -- квазигруппа с нейтральным элементом. 
	\item Полугруппа -- группоид с ассоциативной операцией. 
	\item Моноид -- полугруппа с нейтральным элементом. Примеры моноидов. Пусть $X$ -- некоторое множество, тогда $Y = 2^X$ -- булеан. Рассмотрим следующие структуры на $Y$: $(Y,\cup,\emptyset), (Y,\cap,Y), (Y,\triangle,\emptyset)$. 
	\item Группа -- моноид с существующим обратным элементом для любого элемента моноида. 
	\item Абелева группа -- группа с коммутативной операцией. 
	\item Кольцо -- структура с двумя бинарными операциями (абелева группа по сложению и моноид по умножению) и определенными правой и левой дистрибутивностью умножения относительно сложения. 
	\item Коммутативное кольца -- кольцо с коммутативным умножением. 
	\item Целостное кольцо -- кольцо без делителей нуля (произведение двух ненулевых элементов не равно нулю).
	\item Тело -- кольцо, в котором ненулевые элементы образуют группу по умножению. Иначе, это ассоциативное кольцо с единицей, в котором любой элемент отличный от нуля обратим. Важно, что в любом ассоциативном кольце с единицей $R$ можно рассмотреть все обратимые элементы: $R^{\ast}=\{x\in R \mid \exists y\in R, xy = 1 = yx\}$, тогда $R^{\ast}$ является группой. Если $R^{\ast} = R\setminus\{0\}$, то $R$ -- тело. 
	\item Поле -- коммутативное кольцо, являющееся телом. Характеристикой поля называется то, сколько раз надо сложить с собой нейтральный по сложению, чтобы получить ноль. 
	\item Полукольцо -- кольцо без обратимости сложения. 
	\item Почтикольцо -- группа по сложению и полугруппа по умножению, с определенной правой (или левой) дистрибутивностью умножения относительно сложения. 
	
\end{itemize}

\textbf{Определение}. Левым классом смежности в группе $G$ по подгруппе $H$ называется множество $gH = \{gh\mid h\in H\}$. Аналогично определяется правый класс смежности по подгруппе $H$.

\textbf{Определение}. Нормальной или инвариантной подгруппой $H$ группы $G$ называется такая погруппа $H$, что левый и правый классы смежности по ней совпадают. $N \triangleleft G \Leftrightarrow \forall n\in N, \forall g\in G gng^{-1}\in N$. Очевидно, что, если $G$ -- абелева группа, то всякая её подгруппа нормальна. 

\textbf{Пример}. Подгруппы $\{e\}, G$ группы $G$ всегда нормальные (они называются тривиальными нормальными подгруппами). Если в группе $G$ нет отличных от них нормальных подгрупп, то такая группа $G$ называется \textit{простой} группой. 

\textbf{Пример}. Простейший пример факторкольца -- кольцо классов вычетов по подулю $n$, обозначаемое $\mathbb{Z}\setminus n\mathbb{Z}$. Если кольцо $R = \mathbb{Z}$, идеал кольца $I =  n\mathbb{Z}$, что обозначается как $I \unlhd R$, то $R\setminus I = \mathbb{Z}\setminus n\mathbb{Z}$. 

\textbf{Определение}. Нильпотентом в кольце $R$ называется такое $x\in R$, что $\exists n\in \mathbb{N}$ такое, что $x^n = 0$. В $\mathbb{Z}\setminus n\mathbb{Z}$ нильпотентом будет всякое такое $x\in \mathbb{Z}\setminus n\mathbb{Z}$, которое в степени $x^m = n = 0$. По сути, бесконечно малые некоторого порядка в анализе -- нильпотенты. 

\textbf{Замечание}. В некоммутативном кольце, запись вида $(x+y)(x-y)=x^2+yx-xy-y^2$, где $yx-xy$ -- аддитивный коммутатор, который равен нулю только в случае коммутации $x$ и $y$. То есть, важно осознавать, что применение формул из курса "школьной алгебры" предполагает работу с коммутативным кольцом. Поэтому, например, подставлять матрицы в формулы школьной алгебры нельзя. При том же дифференцировании матриц запись производной произведения функций должна иметь следующий вид: $(fg)'= f'g+fg'$. Забавно то, что даже Г.Лейбниц сделал ощибку, допустив запись вида $(fg)'= f'g+g'f$. 

\textbf{Пример}. Пусть $A$ -- произвольная абелева группа $(A;+)$. Тогда определим операцию $\cdot A\times A\rightarrow A$ следующим образом: $\cdot (x,y) \rightarrow 0$. Полученное кольцо называется \textit{кольцом с нулевым умножением}. 

\textbf{Пример}. Пусть $X$ -- любое множество. Тогда на булеане $R = 2^X$ зададим структуру кольца следующим образом: $\forall Y,Z \subseteq X$ определено сложение $Y+Z = Y\triangle Z$ -- булева сумма и умножение $\cdot$ как пересечение $Y\cdot Z = Y \cap Z$. 

Проверим групповые структуры, образованные этими операциями. 


\textbf{Определение}. Пусть $R$,$S$ -- два кольца. Тогда прямой суммой будет называться следующее множество $R\oplus S = \{(x,y)\mid x\in R, y\in S\}$, а операции сохраняются покомпонентно, то есть, $(x_1,y_1)+(x_2,y_2)$, где $+$ -- сложение в кольце $R\oplus S$, $(x_1,y_1)+(x_2,y_2) = (x_1 + x_2, y_1 + y_2)$, здесь сложение компонент пары -- сложение элементов колец $R$ и $S$. Аналогично определяется умножение: $(x_1,y_1)\cdot(x_2,y_2) = (x_1 \cdot x_2, y_1 \cdot y_2)$. Если оба кольца были, например, ассоциативными, коммутативными кольцами с единицей, то прямая сумма колец тоже будет ассоциативным, коммутативным кольцом с единицей. Подобное верно и для прямой суммы конечного числа колец $R_1\oplus...\oplus R_i$, где $i\in\mathbb{N}$.

Кольца $R,S$ можно вложить в их прямую сумму $R\oplus S$ следующим образом: $R\hookrightarrow R\oplus S$, где $\forall x\in R \mapsto (x,0)\in R\oplus S$ и $S\hookrightarrow R\oplus S$, где $\forall y\in S \mapsto (0,y) \in R\oplus S$. 

Вложение кольца $R$ в кольцо матриц $M(2,R)$ тоже строится аналогично. $R \hookrightarrow M(2,R)$, когда $x \mapsto 
\begin{pmatrix}
x & 0\\
0 & 0
\end{pmatrix}
$. Таким образом мы получим подкольцо без единицы в $M(2,R)$, так как $1 \mapsto e_{11}$. 

Другой вариант вложения $R$ в кольцо матриц $M(2,R)$: $R\hookrightarrow M(2,R)$, когда $x \mapsto 
\begin{pmatrix}
x & 0\\
0 & x
\end{pmatrix}
$. Тогда $R$ подкольцо $M(2,R)$ с единицей. $R \cdot e \leqslant M(2,R)$, где $e = 
\begin{pmatrix}
1 & 0\\
0 & 1
\end{pmatrix}$. 


\textbf{Пример}. $2\mathbb{Z} \leqslant \mathbb{Z}$, где $2\mathbb{Z} = \{n\in\mathbb{Z}\mid n\equiv 0 (mod 2)\}$. 


$R \hookrightarrow R\oplus S \hookleftarrow S$, где $x \mapsto (x,0)$, а $y \mapsto (0,y)$. Тогда $e_R \mapsto (e_R,0)$, а $e_S \mapsto (0,e_S)$, но $e_{R\oplus S} = (e_R,e_S)$. Это пример гомоморфизма колец без сохранения единицы.  Такие гомоморфизмы не являются \textit{унитальными}, то есть, не переводят единицу в единицу. 

\textbf{Пример}. Подкольца в $\mathbb{A}$, где $\mathbb{A}$ -- кольцо целых алгебраических чисел. $x\in\mathbb{C}$ называется целым алгебраическим, если $x$ является корнем некоторого алгебраического уравнения вида $f = t^n + a_{n-1} +\dots + a_1 t + a_0\in \mathbb{Z}[t]$, где $\forall a_i\in\mathbb{Z}$ (целочисленные многочлены со старшим коэффициентом единица). Тогда $f(x) = 0$. Тогда, если для $x,y\in \mathbb{C}$ существуют такие многочлены $f(x) = 0$ и $g(x) = 0$, то существуют многочлены и для $x\cdot y$,$x+y$. $\mathbb{A} \leqslant \mathbb{C}$. 


\textbf{Теорема Гаусса}. $\mathbb{A} \cap \mathbb{Q} = \mathbb{Z}$. 

\textbf{Пример}. Подполей в $\mathbb{Q}$ нет $\Leftrightarrow$ поле $\mathbb{Q}$ -- простое поле. 

\textbf{Определение}. Противоположным кольцом к кольцу $(R;+;\cdot)$ называется кольцо $R^o = R$ такое, что в $(R^o;+;\circ)$ $x\circ y = yx$, $yx\in R$. Кольцо $R$ коммутативно $\Leftrightarrow$ $R=R^o$. То есть, множества носители структуры колец совпадают, но операция умножения в противоположном кольце $R^o$ определяется через операцию умножения в кольце $R$. 

Или, иначе, существует такая биекция $f:R\leftrightarrow R^o$, которая переводит $x\in R$ в $x^o\in R^o$, а операции сохраняются следующим образом: $x^o +y^o = (x+y)^o$ и $x^o \cdot y^o = (yx)^o$. Такое $f$ называется \textit{антиизоморфизмом} колец $R$ и $R^o$. 

\textbf{Определение}. Присоединением единицы к кольцу $R$, где $R$ -- ассоциативное кольцо без единицы (возможно, коммутативное), называется следующая конструкция:
$\mathbb{Z}\oplus R = \{(m,x)\mid m\in\mathbb{Z}, x\in R\}$, $(m,x)+(n,y) = (m+n,x+y)$ и $(m,x)\cdot(n,y) = (mn,my+nx+xy)$. Тогда $(1,0) (m,x) = (m,x) = (m,x)(1,0)$. 

\textbf{Определение}. $L$ называется кольцом Ли, если в ней заданы сложение $+$ и умножение $[,]:L\times L\rightarrow L$, где $x,y \mapsto [x,y]$ -- коммутирование или скобка Ли. Так что при этом выполняются аксиомы кольца и:
\begin{itemize}
	\item  Тождество Якоби $[[x,y],z] + [[y,z],x] + [[z,x],y] = 0$;
	\item Антикоммутативность $[x,x] = 0 \Rightarrow [x,y] = -[y,x]$. 
\end{itemize}

Стандартный пример алгебры Ли -- векторы $\mathbb{R}^3$ относительно векторного умножения и обычного сложения образуют алгебру Ли. 

\textbf{Пример}. Если $R$ -- ассоциативное кольцо, то тогда мы можем определить кольцо $R^{-} = R$, которое как абелева группа по сложению совпадает с $R$, а умножение определяется как коммутирование $[,]: R^{-} \times R^{-} \rightarrow R^{-}$, $x,y \mapsto xy - yx$, то есть, паре элементов сопоставляется их аддитивный коммутатор. Тогда относительно этой операции $R^{-}$ -- кольцо Ли. 

\textbf{Заметка}. Дифференцирование любого объекта образует кольцо Ли. Ассоциативные кольца описывают поведение гомоморфизмов, а кольца Ли описывают поведение дифференцирования. 

\textbf{Определение}. Пусть $S \neq \emptyset$ и $S \subseteq R$, где $R$ -- кольцо. Тогда $S$ называется подкольцом в $R$, если оно является кольцом относительно тех же операций $(R,+,\cdot)$. Это означает, что если $x,y\in S \Rightarrow x-y\in S$ (важно, что не суммы, так как $\mathbb{N}\subseteq\mathbb{Z}$ не является подкольцом, хоть и является замкнутым относительно операций в $\mathbb{Z}$ подмножеством, так как не образует абелеву группу по сложению) и $x,y\in S \Rightarrow x\cdot y\in S$. 

\textbf{Определение}. Если $R$ кольцо с единицей, то $S$ называется \textit{унитальным} (или кольцом с единицей), если $1_R\in S$, то есть, $1_R = 1_S$. 

Чтобы обозначить, что в подкольце $S$ сохраняются алгебраические операции, далее будет использоваться следующим символ $S \leqslant R$. 

\subsection{Гомоморфизмы колец}

\textbf{Определение}. $f: R\rightarrow S$ -- кольцевой гомоморфизм, если: 
\begin{itemize}
	\item $\forall x,y\in R$, выполняется $f(x+y) = f(x) + f(y)$ -- аддитивный гомоморфизм; $\Rightarrow f(0) 0, f(-x) = -f(x)$;
	\item $\forall x,y\in R$, выполняется $f(xy) = f(x)f(y)$ -- мультипликативный гомоморфизм. 
\end{itemize}


\textbf{Определение}. Биективный гомоморфизм называется изоморфизмом. 

\textbf{Определение}. Гомоморфизм в себя называется эндоморфизмом. 

\textbf{Определение}. Эндоморфизм на себя (биективный) называется автоморфизмом. 


Какие еще бывают гомоморфизмы у колец, кроме кольцевых? 

\textbf{Определение}. Лиевский гомоморфизм -- аддитивный гомоморфизм, такой, что $f([x,y]) = [f(x),f(y)]$, $[x,y] = xy - yx$. 

\textbf{Определение}. Йорданов гомоморфизм -- аддитивный гомоморфизм, такой, что $f(xy+yx)=f(x)f(y)+f(y)f(x)$. 

\textbf{Замечание}, которое важно в математическом анализе. Если кольцо $R$ коммутативно и в нем обратима двойка, то есть, $2\in R^*$, то верно, что $xy = \dfrac{1}{2} ((x+y)^2 - x^2 - y^2)$. То есть, в коммутативных кольцах с обратимой двойкой, любой йорданов гомоморфизм является кольцевым гомоморфизмом. 


\textbf{Пример}. Если $R$ кольцо с единицей, то всегда существует такой гомоморфизм $f:\mathbb{Z}\rightarrow R$, который переводит $n \mapsto n \cdot 1_R = \underbrace{1_R + \dots + 1_R}_n$. Если этот гомоморфизм инъективен, то говорят, что характеристика кольца $R$ нулевая: $char(R) = 0$. 
Если $f:\mathbb{Z}\rightarrow R$ не является инъективным гомоморфизмом, то его ядро $Ker(f) = m\mathbb{Z}$, а $charR = m$. Например, $\mathbb{Z}/4\mathbb{Z}$, в котором $2\cdot 2 = 0$, такое кольцо называется \textit{неприведённым кольцом}, так как в нем есть нетривиальные нильпотенты. Или кольцо $\mathbb{Z}/6\mathbb{Z}$, где $2\cdot 3 = 0$, такое кольцо называется \textit{кольцом с делителями нуля}. 

\textbf{Определение}. Коммутативное кольцо $R$ называется областью целостности, если в нём нет нетривиальных делителей нуля (отличных от $0$). 

\textbf{Предложение}. Характеристика произвольной области целостности -- простое число или 0. 


\textbf{Пример}. Вложение кольца в прямую сумму: $i_1: R \hookrightarrow R\oplus S$, где $x \mapsto (x,0)$, $i_2: S\hookrightarrow R\oplus S$, где $y \mapsto (0,y)$. Проекция прямой суммы на кольцо: $pr_1: R\oplus S\rightarrow R$, где $(x,y) \mapsto x$, $pr_2: R\oplus S\rightarrow S$, где $(x,y) \mapsto y$. 

\textbf{Пример}. Гомоморфизм булевого кольца подмножеств $X$ в кольцо функций из $X$ в $\mathbb{F}_2$ (характеристическая функция) является изоморфизмом: $f:2^X \rightarrow \{0,1\}^X$, где $\{0,1\} = \mathbb{F}_2$ -- поле из двух элементов. Дополнение: если $R$ -- произвольное кольцо, а $X$ -- произвольное множество, то сложение функций в $R^X$ определяется поточечно (умножение тоже может так определяться, но это не обязательно): $\forall f,g\in R^X$ верно, что $(f+g)(x) = f(x) + g(x)$ и $(f\cdot g)(x) = f(x)\cdot g(x)$. 


\textbf{Определение}. Идеалом кольца $I \trianglelefteq R$ называется подкольцо, замкнутое относительно умножения на элементы из $R$. Правым идеалом называется такое $I$, что для $\forall i\in I, x\in R$, верно, что $i\cdot r \in I$. Левым идеалом называется такое $I$, что для $\forall i\in I, x\in R$, верно, что $r\cdot i\in I$. Если правый и левый идеалы кольца $R$ совпадают, например, при коммутативности операции $\cdot$, то такой идеал $I$ называют двусторонним идеалом. Двусторонние идеалы в кольцах и алгебрах играют ту же роль, что и нормальные подгруппы в группах. 

\textbf{Свойство}. Ядро гомоморфизма двух колец $f: R\rightarrow S$, $Ker(f) = \{x\in R\mid f(x) = 0 \}$ является идеалом кольца $Ker(f) \trianglelefteq R$. 

\textbf{Свойство}. В кольце $\mathbb{Z}$ все идеалы главные и имеют вид $n\mathbb{Z} = \{nz\mid z\in\mathbb{Z}\}$, где $n\in\mathbb{N}_0$. Главный идеал кольца -- идеал порождённый одним элементом. 

\textbf{Определение}. Пусть $I$ -- двусторонний идеал кольца $R$. Тогда определим на $R$ отношение эквивалентности: $a\sim b \Leftrightarrow a - b\in I$. Класс эквивалентности $a$ обозначается как $[a]$ или $a+I$ -- класс смежности по модулю идеала. Факторкольцо $R/I$ -- множество классов смежности элементов $R$ по модулю $I$, на котором следующим образом определены операции сложения и умножения: $(a+I) + (b+I) = (a+b)+I$ и $(a+I)\cdot(b+I) = ab+I$.  Легко проверить, что эти операции определены корректно, то есть не зависят от выбора конкретного представителя $a$ класса смежности $a+I$. Например, корректность умножения проверяется следующим образом: $a_1 = a+i_1, b_2 = b + i_2, i_{1,2}\in I$. Тогда $a_1 a_2 = (a+i_1)(b+i_2) = ab + i_1 b + ai_2 + i_1 i_2 = ab + I$. В последнем шаге доказательства используется замкнутость идеала $I$ относительно умножения на элемент кольца $R$ (как слева, так и справа) и замкнутость относительно сложения.

\textbf{Теорема о гомоморфизме колец}.  Пусть $f: R\rightarrow S$ -- гомоморфизм колец. Тогда $Imf \cong R/kerf$. Более точно: имеется изоморфизм $\varphi: Imf \rightarrow R/kerf$, ставящий в соответствие каждому элементу $b = f(a) \in Imf$ смежный класс $\pi(a) = a + kerf$. 

\textbf{Доказательство}. Благодаря теореме о гомоморфизме для групп, мы уже знаем, что отображение $\varphi$ является изоморфизмом аддитивных групп. Так что остается только проверить, что этот гомоморфизм сохраняет операцию умножения. Пусть $f(x) = u$, а $f(y) = v$. Тогда $f(xy) = uv$. И $\varphi(uv)  = \pi(xy) = \pi(x)\pi(y) = \varphi(u)\varphi(v)$. 



\subsection{Кольцо матриц}

Пусть $R$ -- ассоциативное кольцо с единицей и $n\in \mathbb{N}$. Тогда $M(n,R)$ -- кольцо квадратных матриц степени $n$ над $R$ (с элементами из $R$). 

\textbf{Определение}. Умножением матриц по Адамару (или по Шуру) называется следующее умножение в $M(n,R)$:

$\begin{pmatrix}
	\alpha_1 & \beta_1\\
	\gamma_1 & \delta_1
\end{pmatrix}
\ast
\begin{pmatrix}
\alpha_2 & \beta_2\\
\gamma_2 & \delta_2
\end{pmatrix}
=
\begin{pmatrix}
	\alpha_1\ast\alpha_2 & \beta_1\ast\beta_2\\
	\gamma_1\ast\gamma_2 & \delta_1\ast\delta_2
\end{pmatrix}$

Это справочная информация, в кольце матриц умножение иное. В кольце матриц произведение -- композиция линейных отображений. Стоит заметить, что исходя из свойств прямой суммы колец, рассмотрение колец матриц с покомпонентным умножением сводится к рассмотрению колец матричных коэффициентов. 

Умножение матриц в кольце  $M(n,R)$:

$\begin{pmatrix}
\alpha_1 & \beta_1\\
\gamma_1 & \delta_1
\end{pmatrix}
\cdot
\begin{pmatrix}
\alpha_2 & \beta_2\\
\gamma_2 & \delta_2
\end{pmatrix}
=
\begin{pmatrix}
\alpha_1\alpha_2 + \beta_1\gamma_2 & \alpha_1\beta_2 + \beta_1\delta_2\\
\gamma_1\alpha_2 + \delta_1\gamma_2 & \gamma_1\beta_2 + \delta_1\delta_2
\end{pmatrix}$

Строки на столбцы умножаются следующим образом: 

$\begin{pmatrix}
x_{1} & \dots & x_{n}
\end{pmatrix}
\cdot
\begin{pmatrix}
x_1\\
\vdots\\
x_n
\end{pmatrix}
=
x_1y_1+x_2y_2+...+x_ny_n \in R$ (результатом является элемент кольца коэффициентов или скаляр). 

$\begin{pmatrix}
x_1\\
\vdots \\
x_n
\end{pmatrix}
\cdot
\begin{pmatrix}
x_{1} & \dots & x_{n}
\end{pmatrix}
= 
 \begin{pmatrix}
x_{11} & \dots & x_{1n} \\
\vdots & \ddots & \vdots \\
x_{n1} & \dots & x_{nn}
\end{pmatrix} \in M(n,R)$ (результатом является элемент кольца матриц над кольцом $R$, такая матрица называется матрицей ранга $1$).

\textbf{Определение}. Единицей кольца матриц называется матрица:
 
$e=
\begin{pmatrix}
1 & \dots & 0 \\
\vdots & \ddots & \vdots \\
0 & \dots & 1

\end{pmatrix}$

\textbf{Определение}. Стандартными матричными единицами называются следующие матрицы: 

$e_{11}=
\begin{pmatrix}
	1 & 0\\
	0 & 0
\end{pmatrix}$
$e_{12}=
\begin{pmatrix}
0 & 1\\
0 & 0
\end{pmatrix}$
$e_{21}=
\begin{pmatrix}
0 & 0\\
1 & 0
\end{pmatrix}$
$e_{22}=
\begin{pmatrix}
0 & 0\\
0 & 1
\end{pmatrix}$


$e_{ij}e_{nu}=\delta_{jn}e_{ik}$, где $\delta_{jn} = 1$, если $j=n$ и $\delta_{jn} = 0$, если $j\neq n$. $\delta_{jn}$ -- символ Кронекера. Отсюда самоочевидно следует, что в $M(n,R)$ есть делители нуля, поэтому это кольцо и не является полем. $e_{11}^2 = e_{11}$, то есть, $e_{11}$ -- идемпотент, а $e_{12}^2 = 0$ -- нильпотент. 


\textbf{Опеределение}. Операция транспонирования это отображение из кольца матриц над $R$ в кольцо матриц над $R^o$. $t:M(n,R)\rightarrow M(n,R^o), x \mapsto x^t$, $(xy)^t = y^t x^t$. 
$\begin{pmatrix}
\alpha & \beta\\
\gamma & \delta
\end{pmatrix}^t
=
\begin{pmatrix}
\alpha^o & \gamma^o \\
\beta^o & \delta^o
\end{pmatrix}$, если же $R$ -- коммутативное кольцо, то 
$\begin{pmatrix}
\alpha & \beta\\
\gamma & \delta
\end{pmatrix}^t
=
\begin{pmatrix}
\alpha & \gamma \\
\beta & \delta
\end{pmatrix}$. 

\subsection{Кольцо многочленов}

Пусть $R$ -- некоторое ассоциативное, коммутативное кольцо с единицей. Тогда $(R[t];+;\cdot)$ -- кольцо многочленов с коэффициентами из кольца $R$. Элементами кольца $R[t]$ будут $f=a_n t^n +...+a_1 t + a_0$, где $a_i\in R$. Умножение же в кольце $R[t]$ называется свёрткой. 

\textbf{Замечание}. Следующая структура на множестве многочленов с коэффициентами в $R$ кольцом не будет -- $(R[t];+;\circ)$, где $+$ -- обычное сложение многочленов, а $\circ$ -- композиция многочленов вида $f\circ g = f(g)$. Например, $t^2 \circ (t+1) = (t+1)^2$, а $(t+1)\circ t^2 = t^2 + 1$. Такое умножение ассоциативно, а нейтральным элементом относительно умножения будет $t$. Однако с дистрибутивностью всё так хорошо: $(f_1+f_2)\circ g = f_1(g) + f_2(g) = f_1 \circ g + f_2 \circ g$ и $g\circ(f_1+f_2) = g\circ f_1 + g\circ f_2$. Вторая дистрибутивность почти никогда не выполняется, только в некоторых полях ненулевой характеристики.  

\chapter{Линейная алгебра и аналитическая геометрия}


\section{Аналитическая геометрия}

Угловой коэффициент прямой. 

\section{Линейные отображения}

\textbf{Определение}. Линейным отображением векторного пространства $f: L_K \rightarrow M_K$, где оба векторных пространства заданы над одним полем $K$, называется такая функция $f$, что выполнены следующие условия: 

\begin{itemize}
	\item $f(a+b) = f(a) +f(b)$ 
	\item $f(\alpha b) = \alpha f(b)$, где $\alpha \in K$
\end{itemize}
Что верно для любых векторов из пространства $L_K$ и элементов поля $K$. 

Если определить операции сложения и умножения на скаляр из поля $K$ как: 

\begin{itemize}
	\item $(f+g)x = f(x) + g(x)$, для $\forall x\in L_K$. Такая операция называется \textit{поточечной суммой функций}. Сложение для образов определено, так как оно определено в векторном пространстве $M_K$. Поэтому образом суммы двух функций будет просто сумма образов, в смысле векторной суммы. 
	\item $(\alpha f)(x) = \alpha f(x)$, для $\alpha \in K$ и $\forall x\in L_K$, что верно, так как умножение на элементы поля определено для элементов пространства $M_K$
\end{itemize}

То множество всех таких линейных отображений из $L_K$ в $M_K$ образует векторное пространство над полем $K$, которое обозначается как $Hom(L_K,M_K)$. То есть, на множестве таких функций строится структура группы.  

Если пространства $L_K$ и $M_K$ совпадают, то такое линейное отображение называется \textit{линейным преобразованием} пространства $L_K$ или \textit{линейным оператором}. На пространстве $Hom(L_K,L_K)$ явно строится структура кольца. Умножение на $Hom(L_K,L_K)$ определяется как:




Моноид по умножению на $Hom(L_K,L_K)$ содержит подгруппу, изоморфную $GL(L_K)$, которая будет определена ниже. 

\textbf{Определение}. Векторные пространства, между которыми есть биективное линейное отображение, называются \textit{изоморфными}, а само это отображение называется \textit{изоморфизмом} векторных пространств. Изоморфизм векторного пространства с самим собой, построенный посредством тождественного отображения каждого вектора в себя, называется \textit{автоморфизмом}. Автоморфизмы векторно пространства $L_K$ образуют \textit{группу преобразований} этого векторного пространства. Эта группа обозначается как $GL(L_K)$ и называется \textit{полной линейной группой} векторного пространства $L_K$. По сути, полная линейная группа -- группа обратимых линейных преобразований $f: L_K \rightarrow L_K$. Роль групповой операции играет композиция линейных преобразований (читайте, композиция отображений). Очевидно, что не все линейные отображения из $Hom(L_K,L_K)$ являются автоморфизмами: можно отобразить всё пространство в нулевое подпространство и так далее. 

Если у группы $GL(L_K)$ ограничить порядок неким $n\in \mathbb{N}$, то полученная группа будет состоять из всех тех линейных преобразований, которым биективно соответствует некоторая матрица $n\times n$ в $M_n(K)$, где $K$ -- поле, над которым построено наше векторное пространство $L_K$, а обозначаться такая группа будет как $GL(n,L_K)$.

Ещё раз: у нас есть векторное пространство $L_K$ над некоторым полем $K$, кольцо матриц $М(n,L_K)$, составленных из элементов этого поля и биективно соответствующих различным \textit{линейным отображениям} нашего пространства, которые образуют кольцо $GL(L_K)$, а также \textit{двойственное векторное пространство}, которое будет определено ниже. Все эти структуры органично связаны друг с другом: задав явно векторное пространство $L_K$ мы тут же получаем и полную линейную группу, и двойственное пространство, и различные векторные пространства, изоморфные нашему пространству $L_K$ (на самом деле, мы получаем все гомоморфные линейные пространства, а изоморфные уже как частный случай). 

Вообще, всякую матрицу порядка $n\times n$, составленную из элементов поля $K$, над которым построено наше векторное пространство $L_K$, можно рассматривать как линейное преобразование, действующее на \textit{арифметическом векторном пространстве} $K^n$. Арифметическое векторное пространство состоит из наборов элементов поля $K$, каждый набор имеет длину $n$ -- значит, что он состоит из $n$ элементов поля. Это хорошо знакомые нам $n$-мерные векторы-столбцы или векторы-строки. Поэтому $GL(n,L_K) = GL(K^n)$. Поэтому значительно проще работать с арифметическим пространством и группой $GL(K^n)$ -- и там, и там мы сводим всё к арифметическим операциям над элементами поля $K$. 

Чтобы явно построить матрицы из $GL(K^n)$, которые бы задавали некоторые линейные преобразования нашего пространства $L_K$, нужно выбрать некоторый базис в пространстве $L_K$ и сопоставить каждому линейному преобразованию $f: L_K \rightarrow L_K$ матрицу в $GL(K^n)$, которая бы состояла из компонент оператора $f$ в фиксированном базисе пространства $L_K$.  При этом обратимому оператору будет отвечать \textit{невырожденная матрица}, и мы получаем взаимно однозначное соответствие между группами $GL(L_K)$ и $GL(n, L_K)$ (это соответствие в действительности является изоморфизмом данных групп).

\textbf{Утверждение}. Всякое конечномерное векторное пространство $L_K$ размерности $n$ изоморфно арифметическому векторному пространству $K^n$. 





\textbf{Пример}. Важным примером векторного пространства является проство $K_K$ -- аддитивной группы поля над самим полем. Элементами такого пространства являются просто элементы поля, с определённой на них операцией сложения и умножения на скаляры -- другие элементы поля $K$. Такое поле также можно назвать полем матриц размера $1\times 1$ и записать как $M_1 (K) = K$. Например, множество вещественных чисел образует одномерное векторное пространство над самим собой, в котором все векторы (вещественные числа) пропорциональны -- \textit{коллинеарны}, а базисом такого пространства $\mathbb{R_{\mathbb{R}}}$ будет любая точка на вещественной прямой. Умножение же матрица больших размеров уже некоммутативно, что объясняет невозможность построения на них структуры поля -- только кольца. 


\section{Сравнения по модулю}

\textbf{Определение}. Два числа $a$ и $b$ называются сравнимыми по модулю $n$, если при делении на $n$ они дают одинаковые остатки. Записывается это так: $a \equiv b (mod n)$, а читается как "$a$ сравнимо с $b$ по модулю $n$. 

\textbf{Пример}. Группа матриц $SO(n)\subset SL(n)\subset GL(n)$ и $O(n)$, где операции определены на $detM$, то есть, определителях матриц. 


\chapter{Основы общей топологии}
\section{Топология в множестве}
\subsection{Базовые понятия}
\textbf{Определение топологического пространства}. Пусть $X$ -- некоторое множество. Рассмотрим такой набор его подмножеств $\Omega$, для которого выполнены следующие аксиомы \textbf{топологического пространства}: 
\begin{itemize}
	\item Объединение любого семейства множеств, принадлежащих набору $\Omega$, также принадлежит совокупности $\Omega$;
	\item Пересечение любого конечного семейства множеств, принадлежащих $\Omega$, также принадлежит $\Omega$;
	\item Пустое множество $\emptyset$ и всё $X$ принадлежат $\Omega$. 
\end{itemize}

Такое $\Omega$ называется \textbf{топологией} или \textbf{топологической структурой} в множестве $X$. Такое множество $X$, с выделенной топологией $\Omega$, обозначается $(X,\Omega)$ и называется топологическим пространством. 
Элементы множества $X$ -- точки, а элементы $\Omega$ -- открытые множества пространства $(X,\Omega)$.

\textbf{Примеры}
\\ 
\textit{Дискретное пространство} -- множество, в котором $\Omega$ является совокупностью всех подмножеств $X$. Проверьте, что аксиомы топологического пространства выполняются.
\\ 
\textit{Антидискретное пространство} -- множество, в котором $\Omega$ состоит только из $X$ и $\emptyset$. 
\\
\textit{Вещественная прямая} -- $\mathbb{R} = X$ и $\Omega$, являющаяся совокупностью объединений семейств интервалов вида $(a,b)$, где $a,b\in\mathbb{R}$. Такое пространство называется вещественной прямой, а топологическую структуру такого вида называют канонической или стандартной топологией на $\mathbb{R}$. 
\\
\textit{Топология Серпинского} -- на произвольном двухэлементном множестве $X = \{0,1\}$ задано $\Omega = \{\emptyset, \{0\}\{0,1\}\}$. 

\textbf{Свойства замкнутых множеств}:
\begin{itemize}
	\item пересечение любого (даже бесконечного) набора замкнутых множеств замкнуто. Это напрямую следует из определения замкнутого множества как дополнения открытого и из формулы \ref{proof2}, бывшей в первой главе; 
	\item объединение любого \textit{конечного} набора замкнутых множеств замкнуто;
	\item пустое множество $\emptyset$ и всё множество носитель структуры топологического пространства, то есть $X$, -- замкнутые множества.
\end{itemize}

Обратите внимание, что в топологическом пространстве существуют множества, которые не являются ни открытыми, ни замкнутыми. \textit{Важно понимать, что если говорят о "неоткрытом" множестве, то не обязательно предполагают замкнутое множество.} Очевидным примером такого множества в $\mathbb{R}$ является полуинтервал вида $[a,b)$, который представим как пересечение открытых или объединение замкнутых множеств. 

Привидите примеры таких множеств в произвольном топологическом пространстве. 

Главное отличие открытых множеств от замкнутых состоит в том свойстве, что пересечение бесконечного числа открытых множеств не обязательно замкнуто, а объединение бесконечного числа замкнутых множеств замкнуто. 

Очевидно, что в дискретном пространстве замкнутыми являются все множества, а в антидискретном только пустое множество и $X$. 

\textbf{Окрестностью} точки топологического пространства называется любое открытое множество, которое содержит эту точку (как элемент). Отсюда самоочевидно следует, что всякая точка входит в открытое множество с некоторой своей окрестностью. 
 
\subsection{Понятие базы топологии}

\textbf{Базой топологии} называют некоторую совокупность открытых множеств, такую что всякое непустое открытое множество, принадлежащее топологии, представимо в виде объединения элементов базы. Очевидно, что различные топологические структуры не могут иметь одну базу -- по базе можно восстановить топологическую структуру. Базы, задающие одну и ту же топологическую структуру называются \textit{эквивалентными}.

Например, база дискретного пространства -- множество всех одноточечных подмножеств (она минимальна). В антидискретном пространстве минимальная база состоит из всего пространства, что тоже очевидно. 

\textbf{Утверждение}. Совокупность открытых множеств $\varSigma$ называется базой топологии $\Omega$, тогда и только тогда, когда для всякого множества $\forall U\in\Omega$ и всякой точки $\forall x\in U$ верно, что $\exists V \in \varSigma$ такой, что $x\in V\subset U$. 

\textbf{Доказательство}. В одну сторону. Допустим, что $\varSigma$ -- база топологии $\Omega$ и $U\in\Omega$. Представим множество $U$ в виде объединения элементов базы. Тогда $\forall x\in U$ покрыто некоторым элементом базы $\varSigma$. Такое множество и можно взять в качестве $V$. В обратную сторону. Допустим, что для $\forall U\in \Omega$ и $\forall x\in U$ существует такое $\exists V\in\varSigma$, что $x\in V\subset U$, тогда $\varSigma$ -- база топологии $\Omega$. Для этого нужно убедиться, что $\forall U\in\Omega$ представляется в виде объединения элементов из $\varSigma$. Для каждой точки $x\in U$ выберем такое $V_x\in \varSigma$, пользуясь предположением, что $x\in V_x\subset U$ и рассмотрим $\bigcup_{x\in U} V_x$. Очевидно, что $\bigcup_{x\in U} V_x \subset U$, так как $V_x\subset U$ $\forall x\in U$. Аналогично получаем, что $U\subset \bigcup_{x\in U} V_x$ $\Rightarrow$ $U =\bigcup_{x\in U} V_x$.

\textbf{Утверждение}. Совокупность $\varSigma$ подмножеств $X$ является базой некоторой топологии $\Omega$ на $X$, тогда и только тогда, когда $X$ является объединением элементов этой совокупности $\varSigma$ и пересечение любых двух множеств из $\varSigma$ представляется как объединение множеств из $\varSigma$. 



Примеры баз топологической структуры на $\mathbb{R}$:
\begin{itemize}
	\item База, состоящая из всевозможных открытых кругов;
	\item База, состоящая из всевозможных открытых квадратов, стороны которых параллельны координатным осям и задаются неравенствами вида $max\{|x-a|,|y-b|\}<r$;
	\item База, состоящая из всевозможных открытых квадратов, стороны которых параллельны биссектрисам координатных углов. Они задаются неравенствами вида $|x-a|+|y-b|<r$.
\end{itemize}

Интересный пример с базой канонической топологии $\mathbb{R}$ -- в ней нет минимальной базы. Доказательство самоочевидно следует из полноты $\mathbb{R}$. 

\subsection{Грубые и тонкие топологии}

Пусть $\Omega_1$ и $\Omega_2$ -- топологии на множестве $X$. Тогда, если $\Omega_1 \subset \Omega_2$, то $\Omega_1$ называют более грубой топологией, чем $\Omega_2$, а $\Omega_2$ более тонкой, чем $\Omega_1$ топологией. 

Антидискретная топология -- самая грубая топология, а дискретная -- самая тонкая топология (в произвольном множестве). 

\subsection{Непрерывное отображение}

\textbf{Определение}. Непрерывным отображением называется такое отображение $f: X\rightarrow Y$, где $(X,\Omega_1)$ и $(Y,\Omega_2)$ -- топологические пространства, такое, что всякий прообраз открытого $ A \subset Y$ , $A\in\Omega_2$ множества открыт в $X$: $f^{-1}(A)\subset X$ , $f^{-1}(A)\in\Omega_1$. Очевидно, если такой прообраз у $A$ при отображении $f$ вообще есть. 

\textbf{Эквивалентные формулировки:} 
\begin{itemize}
	\item Прообраз всякого открытого множества открыт;
	\item Прообраз всякого замкнутого множества замкнут;
	\item Прообраз каждой окрестности точки области значений отображения является окрестностью соответствующей точки области определения;
	\item Образ замыкания любого множества содержится в замыкании образа этого множества;
	\item Замыкание прообраза любого множества содержится в прообразе замыкания.
\end{itemize}

Существование непрерывных отображений между пространствами, позволяет «переносить» свойства одного пространства в другое: например, непрерывный образ компактного пространства также является компактным. Это теорема, которая будет доказана позже (совместно с определение компактного пространства). 

Когда говорят об отображениях между топологическими пространствами, то всегда предполагают по-умолчанию, что это отображение непрерывное, если, конечно, не сказано обратного. 

\textbf{Определение}. Гомеоморфизмом называется такое непрерывное отображение $f: X\rightarrow Y$, которое является биективным, а обратное к нему $f^{-1}$ также непрерывно. 

Гомеоморфные топологические пространства считаются эквивалентными, то есть, гомеоморфизм порождает отношение эквивалентности в классе топологических пространств. Свойства топологических пространств, сохраняемые при гомеоморфизмах называются \textit{топологическими инвариантами}. 


\textbf{Определение}. Индуцированной топологией на подмножестве $A\subset X$ называется топология $\Omega_A$ элементами которой являются всевозможные пересечения множества $A$ с элементами топологии $\Omega_1$, заданной на $X$. 

Отображение $f: X\rightarrow Y$ непрерывное на $X$, будет непрерывно и на любом его подмножестве, в смысле индуцированной топологии на этом подмножестве. 


\textbf{Пример}. Непрерывное отображение, как непрерывная функция в математическом анализе. 


\subsection{Определение метрического пространства}

Функция $\rho:X\times X \rightarrow \mathbb{R}_+$, где $\mathbb{R}_+ = \{r\in\mathbb{R}$ $|$ $r \geqslant 0\}$ называется \textit{метрикой}, если выполняются следующие \textit{аксиомы метрического пространства}:
\begin{itemize}
	\item $\rho(x_1,x_2) = 0 \Leftrightarrow x_1=x_2$ (невырожденность);
	\item $\rho(x_1,x_2) = \rho(x_2,x_1)$, для $\forall x_i\in X$ (симметричность);
	\item $\rho(x_1,x_2) + \rho(x_2,x_3) \geqslant \rho(x_1,x_3)$, для $\forall x_i\in X$. Эта аксиома называется \textit{неравенством треугольника}, попробуйте объяснить почему. 
\end{itemize} 

Пара $(X,\rho)$, где $X$ -- произвольное множество, а $\rho$ -- функция, удовлетворяющая вышеперечисленным аксиомам, называется \textit{метрическим пространством}. 

По сути, метрика -- математическая абстракция, отвечающая интуитивному представлению о "расстоянии". Строгое определение метрического пространства дал Морис Фреше в 1906-м году. 

Проверьте, выполняются ли эти условия для следующих функций: 
\begin{itemize}
	\item $\rho: X\times X \rightarrow\mathbb{R}_+: \rho(x_1,x_2) \rightarrow 0$, если $x_1=x_2$ и $\rho(x_1,x_2) \rightarrow 1$, если $x_1\neq x_2$;
	\item $\rho:\mathbb{R} \times \mathbb{R} \rightarrow \mathbb{R}_+$, где $\rho(r_1,r_2) \rightarrow |r_1 - r_2|$;
	\item $\rho:\mathbb{R}^n \times\mathbb{R}^n\rightarrow \mathbb{R}_+$, где $\rho(x,y) \rightarrow \sum_{i=1}^{n}\sqrt{(x_i - y_i)^2}$. Предыдущая метрика -- частный случай этой метрики. 
\end{itemize}

\subsection{Последовательности Коши} 
\textbf{Определение}. Пусть $x\in M$ -- точка в метрическом пространстве $M$. Открытый $\varepsilon$-шар $B_{\varepsilon}(x)$ с центром в точке $x$, то есть множество таких точек пространства, которые отстоят от $x$ менее чем на $\varepsilon$. 

\begin{equation}
B_{\varepsilon}(x)=\{y\in M\mid \rho(x,y) < \varepsilon\}
\end{equation}

\textbf{Определение}. Пусть $M$ -- метрическое пространство. Последовательность $\{\alpha_i\}$ точек из $M$ называется \textit{последовательностью Коши}, если для $\forall \varepsilon > 0$, все элементы последовательности $\{\alpha_i\}$, кроме конечного числа, содержатся в этом $\varepsilon$-шаре. 

\textbf{Определение}. Последовательности Коши $\{\alpha_i\}$ и $\{\beta_i\}$ называются \textit{эквивалентными}, если последовательность вида $\alpha_0, \beta_0, \alpha_1, \beta_1,...$ является последовательностью Коши. 

\textbf{Свойство}. Всякая сходящаяся последовательность метрического пространства является последовательностью Коши, но не всякая последовательность Коши сходится к какому-то элементу из своего метрического пространства. 

\textbf{Определение}. Пусть $M$ -- метрическое пространство. Последовательность $\{\alpha_i\}$ точек из $M$ \textit{сходится} к $x\in M$, если в $\forall\epsilon$-шаре $B_{\epsilon}(x)$ содержатся все члены последовательности $\{\alpha_i\}$, кроме конечного числа. Тогда $x$ называется пределом последовательности $\{\alpha_i\}$. 

\textbf{Определение}. Метрическое пространство $M$ называется \textit{полным}, если у любой последовательности Коши есть предел. Например, $\mathbb{R}$ --полное метрическое пространство, что означает, что всякая последовательность Коши в нём имеет предел. Проверьте это утверждение самостоятельно. 

\textbf{Пример}. Пусть $\mathbb{Q}$ -- метрическое подпространство $\mathbb{R}$. Тогда очевидно, что последовательность Коши $\{\alpha_i\}$, где $\alpha_i\in\mathbb{Q}$, а $\{\alpha_i\} \subset \mathbb{R}$, такая, что пределом является $x=\sqrt{2}$,  в $\mathbb{Q}$ не имеет предела. 

\textbf{Свойства последовательностей Коши}

\begin{itemize}
	\item Всякая подпоследовательность последовательности Коши является последовательностью Коши. Последовательность Коши эквивалентна любой своей подпоследовательности; 
	\item Если переставить произвольным образом элементы последовательности Коши, то получится эквивалентная изначальной последовательность Коши;
	\item Предел единственен, если существует. 
\end{itemize}

\subsection{Пополнение}

\textbf{Определение}. Диаметр множества $X\subset M$ есть $sup_{x,y\in X}$ $\rho(x,y)$.
\\
\textbf{Утверждение}. Диаметр $\epsilon$-шара не больше $2\epsilon$.

\textbf{Следствие}. Очевидным следствием из этого является, что для любой последовательности Коши $\{\alpha_i\}$ и $\forall x\in M$, последовательность вещественных чисел $\{\rho(x,\alpha_i)\}$ -- последовательность Коши в $\mathbb{R}$. 

\textbf{Следствие}. Для любых последовательностей Коши $\{\alpha_i\}$ и $\{\beta_i\}$ последовательность $\{\rho(\alpha_i,\beta_i)\}$ -- последовательность Коши. Проверьте это. 

\textbf{Следствие}. Это задает метрику на классах эквивалентности последовательностей Коши: 

\begin{equation}
\rho(a_i,b_i) = \lim_{i\rightarrow\infty}| a_i - b_i |
\end{equation}

\textbf{Определение}. Множество классов эквивалентности последовательностей Коши в $M$ с метрикой, определённой выше, называется \textit{пополнением} $M$. 

\textbf{Теорема}. Пополнение является полным метрическим пространством. Проверьте аксиомы самостоятельно. 

\subsection{Метрики на абелевых группах}

\textbf{Определение}. Пусть $M,N$ -- метрические пространства. Вложение $f:M \hookrightarrow N$ называется \textit{изометрическим вложением}, если $f$ сохраняет расстояния для $\forall x,y\in M$: 

\begin{equation}
\rho_M(x,y) = \rho_N(f(x),f(y))
\end{equation}

Изометрические пространства -- пространства между которыми существует биекция, сохраняющая расстояния. 

\textbf{Определение}. Пусть $(G,+)$ -- абелева группа, а $\rho$ -- метрика на множестве носителе структуры группы $G$. Тогда $(G,+,\rho)$ -- метрическая группа, если операция взятия обратного элемента $x\rightarrow -x$ и операция $x \rightarrow x+g$, $\forall g\in G$ -- есть изометрии. В таких случаях говорят о \textit{согласованности метрики с групповой структурой}, или что метрика \textit{инвариантна}.

"Группа с инвариантной метрикой действует сама на себе изометриями". 


\subsection{Понятие нормы}
Норму можно рассматривать как обобщение на многомерный случайн понятия \textit{абсолютной величины числа}. 


\subsection{Задание инвариантных метрик через норму}


\textbf{Определение}. Топологическая группа -- множество $G$, такое, что:
\begin{itemize}
	\item $G$ -- группа $(G;\cdot)$
	\item $G$ -- топологическое пространство 
	\item Групповая операция непрерывна: для $\forall g_1, g_2\in G: g_1\cdot g_2 = g_3$ и для $\forall W$, где $W$ -- некоторая окрестность $g_3$, $\exists U,V$, окрестности $g_1,g_2$, такие, что $UV\subset W$. 
\end{itemize}

\chapter{Математический анализ}
\textbf{Определение}. Число $a\in\mathbb{R}$ называется пределом последовательности $\{x_n\}$, если $\{x_n-a\}$ -- б.м.п. Последовательность $\{x_n\}$ называется сходящейся к числу $a$. Пишут: $\lim_{n\rightarrow\infty} x_n = a$. 

\textbf{Пример}. $\dfrac{n+1}{n}\rightarrow 1$, так как $| x_n - 1 | = \dfrac{1}{n}$, а последовательность $\{\dfrac{1}{n}\}^{\infty}_{n=1}$ -- б.м.п.

Из определения б.м.п. имеем: $a = \lim_{n\rightarrow\infty}x_n$, если для $\forall\epsilon>0$ $\exists N\in\mathbb{N}$, такое что для $\forall n\geq N: | x_n - a | <\epsilon$. 

\textbf{Определение}. $\epsilon$-окрестностью называется такой интервал на вещественной оси $x_n\in (a-\epsilon,a+\epsilon)$, такой, что $| x_n - a | < \epsilon \Leftrightarrow -\epsilon<x_n - a<\epsilon \Leftrightarrow a-\epsilon<x_n<a+\epsilon$. Обозначим такую окрестность как $B_{\epsilon}(a)$, то есть, это открытый одномерный шар радиуса $\epsilon$. 

\textbf{Определение}. Дополнительное определение предела последовательности. Точка $a$  называется пределом последовательности $\{x_n\}$, если в $\forall\epsilon$-окрестности точки $a$ содержатся все элементы последовательности, начиная с некоторого $N\in\mathbb{N}$. $N$ зависит от $\epsilon$, то есть, существует некая функция $f:N\rightarrow N(\epsilon)$.

\textbf{Определение}. Последовательность $x_n$ стремится к $+ \infty(- \infty)$, если $\forall A > 0$ $\exists N = N(A)\in \mathbb{N}$, такое, что $\forall n \geq N$: $x_n > A$ $(x_n <-A)$. Пишут: $\lim_{n\rightarrow\infty}x_n = + \infty (- \infty)$. 

Отсюда следует, что если $\lim_{n\rightarrow\infty}x_n = \infty$ и $\forall A > 0$ $\exists N = N(A)\in\mathbb{N}$, такое, что для $\forall n\geq N$: $| x_n | > A$. $\Leftrightarrow \{x_n\}$ -- б.б.п. 

Две фундаментальные последовательности эквивалентны $\{a_n\}\sim\{b_n\}$ тогда и только тогда, когда их разность $a_n - b_n \rightarrow 0$ стремится к нулю. 
$\mathbb{R}$ -- фактор-множество таких фундаментальных последовательностей.
\\
\textbf{Определение}. Сумму $s_n = \sum^{n}_{k=1}a_k$ называют частичной суммой ряда (или $n$-ой частичной суммой ряда).
\\
\textbf{Определение}. Сходящимся рядом называется такой ряд, последовательность $\{s_n\}$ частичных сумм которого сходится. Если последовательность $\{s_n\}$ не имеет предела, то ряд называется расходящимся. 
\\
\textbf{Определение}. Предел последовательности частичных сумм ряда $\lim_{n\rightarrow\infty}s_n = s$, если он существует, называется суммой ряда. 

\begin{equation}
\sum_{n=1}^{\infty} a_n = s
\end{equation}

Таким образом, под рядом мы понимаем пару последовательностей $(\{a_n\},\{s_n\})$, связанных соотношением: $\forall n\in\mathbb{N}$ верно, что $\sum_{k=1}^{n} a_n = s_n$. 


\chapter{Дифференциальные и интегральные уравнения}
 

\chapter{Теория измерений и общая теория меры}

\begin{flushright}
	Измерить всё, что измеримо, и сделать измеримым всё то, что таковым ещё не является.
	 
	Галилей
\end{flushright}
\chapter{Базовая комбинаторика и дискретная математика}

\subsection{Основные понятия элементарной комбинаторики}

Комбинаторные конфигурации: 
\begin{itemize}
	\item Схема выбора без возвращения -- когда выбирается первый элемент, затем фиксируется и откладывается. Второй элемент выбирается без участия первого, затем фиксируется и откладывается и так далее;
	\item Схема выбора с возвращением -- когда выбирается первый элемент, затем фиксируется и возвращается обратно в исходное множество. Затем выбирается второй элемент, фиксируется и возвращается обратно в исходное множество и так далее. 
\end{itemize}

\textbf{Определение}. Перестановка -- любой упорядоченный набор $n$-элементного множества. 
Перестановка с повторениями вычисляется как $P_n = n!$. Перестановка без повторений вычисляется как $P^{n_1,n_2,...,n_k}_n = \frac{n!}{n_{1}!n_{2}!...n_{k}!}$, где $n_k$ -- число размещений в наборе $k$-го элемента перестановки. Например, число перестановок элементов в трехэлементном множестве $\{1;2;3\}$: $\{1;2;3\},$ $\{3;1;2\},$ $\{2;3;1\},$ $\{3;2;1\}, \{2;1;3\}, \{1;3;2\}$. В случае с перестановками, в которых разрешены повторения элементов, всё выклядит вот так: $\{1;1;2\},$ $\{1;2;1\},$ $\{2;1;1\}$. 

\textbf{Определение}. Размещение -- любое упорядоченное k-элементное подмножество исходного n-элементного множества. Размещение без повторений вычисляется как $A^k_n = \frac{n!}{(n-k)!}$. Размещение с повторениями вычисляется как $B^k_n = n^k$. Например, упорядоченные двухэлементные подмножества множества $\{1;2;3\}$: $\{1;2\}, \{1;3\}, \{2;3\}, \{2;1\}, \{3;1\}, \{3;2\}$, а если считать с повторами, то просто добавим к уже имеющимся $\{1;1\}$, $\{2;2\}$, $\{3;3\}$. 

\textbf{Определение}. Сочетание -- любое k-элементное подмножество исходного n-элементного множества (неупорядоченное). Сочетание без повторений вычисляется как $C^k_n = \frac{n!}{k!(n-k)!}$. Сочетание с повторениями вычисляется как $S_n^k = C_{n+m-1}^k$. Например, двухэлементные подмножества без повторов в множестве $\{1;2;3\}$: $\{1;2\}, \{1;3\}, \{2;3\}$, а если считать с повторами, то просто добавим к уже имеющимся $\{1;1\}$, $\{2;2\}$, $\{3;3\}$. 

\chapter{Теория вероятностей и математическая статистика}
\subsection{Классическая теория вероятностей}

Теория вероятностей: изестно общее поведение объекта $\longrightarrow$ предсказываем частное поведение объекта.
 
Математическая статистика: известно частное поведение объекта $\longrightarrow$ необходимо построить общую предсказательную модель поведения данного объекта. 

\subsection{Геометрическая вероятность}
Задачи на случаение бросание точки в подпространство. 

Задача Бернштейна о тетраэдре. 

\subsection{Аксиоматическая теория вероятностей}

\subsection{Математическая статистика}


\chapter{Проективная и аффинная геометрия}
\chapter{Неевклидовы геометрии}


\end{document}
